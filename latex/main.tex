\documentclass[12pt,a4paper]{book}

%Uporabljeni paketi
\usepackage[utf8]{inputenc}
\usepackage{cmap}
\usepackage{type1ec}
\usepackage[T1]{fontenc}
\usepackage{fancyhdr}
\usepackage{graphicx,epsfig}
\usepackage[slovene]{babel}
\usepackage{cite}
\usepackage[pdftex,colorlinks,citecolor=black,filecolor=black,linkcolor=black,urlcolor=black,pagebackref]{hyperref}
\usepackage{tikz}
%Velikost strani - dvostransko
\oddsidemargin 1.4cm
\evensidemargin 0.35cm
\textwidth 14cm
\topmargin 0.26cm
\headheight 0.6cm
\headsep 1.5cm
\textheight 20cm

%Nastavitev glave in repa strani
\pagestyle{fancy}
\fancyhead{}
\renewcommand{\chaptermark}[1]{\markboth{\textsf{Poglavje \thechapter:\ #1}}{}}
\renewcommand{\sectionmark}[1]{\markright{\textsf{\thesection\  #1}}{}}
\fancyhead[RE]{\leftmark}
\fancyhead[LO]{\rightmark}
\fancyhead[LE,RO]{\thepage}
\fancyfoot{}
\renewcommand{\headrulewidth}{0.0pt}
\renewcommand{\footrulewidth}{0.0pt}

\newcommand{\gnuplot}{\textbf{gnuplot}}
\newcommand{\pgfname}{\textsc{pgf}}
\newcommand{\tikzname}{Ti\emph{k}Z}
\usepackage[backend=biber,style=apa]{biblatex}
\addbibresource{references.bib} 
\begin{document}
\thispagestyle{empty} 
\begin{center}
{\large 
UNIVERZA V LJUBLJANI\\
FAKULTETA ZA RAČUNALNIŠTVO IN INFORMATIKO\\
}

\vspace{3cm}
{\LARGE Svit Spindler}\\

\vspace{2cm}
\textsc{\textbf{\LARGE 
Hiter inženiring pri razvoju programske opreme\\ 
}}

\vspace{2cm}
{ DIPLOMSKO DELO}\\
{ NA UNIVERZITETNEM ŠTUDIJU\\
}

\vspace{2cm} 
{\Large Mentor: izr. prof. dr. Dejan Lavbič}

\vfill
{\Large Ljubljana, 2024}
\end{center}

\newpage

%********************************************

% stran 3 med uvodnimi listi
\thispagestyle{empty}

Namesto te strani {\bf vstavite} original izdane teme diplomskega dela s podpisom mentorja in dekana ter \v zigom fakultete, ki ga diplomant
dvigne v študent\-skem referatu,  preden odda izdelek v vezavo!

\newpage

\newpage

\ \thispagestyle{empty}

\newpage

%********************************************

% stran 2 med uvodnimi listi
\thispagestyle{empty}

\vspace*{5cm}
{\small \noindent
To diplomsko delo je ponujeno pod licenco \textit{Creative Commons Priznanje avtorstva-Deljenje pod enakimi pogoji 2.5 Slovenija}
ali (po želji) novejšo različico.
To pomeni, da se tako besedilo, slike, grafi in druge sestavine dela kot tudi rezultati diplomskega dela lahko prosto distribuirajo,
reproducirajo, uporabljajo, dajejo v najem, priobčujejo javnosti in predelujejo, pod pogojem, da se jasno in vidno navede avtorja in naslov tega
dela in da se v primeru spremembe, preoblikovanja ali uporabe tega dela v svojem delu, lahko distribuira predelava le pod
licenco, ki je enaka tej.
Podrobnosti licence so dostopne na spletni strani \url{http://creativecommons.si/} ali na Inštitutu za
intelektualno lastnino, Streliška 1, 1000 Ljubljana.

\begin{center}% 0.66 / 0.89 = 0.741573033707865

\end{center}
}

\vspace*{1.5cm}
{\small \noindent
Izvorna koda diplomskega dela, njenih rezultatov in v ta namen razvite programske opreme je ponujena pod GNU General Public License,
različica 3 ali (po želji) novejšo različico. To pomeni, da se lahko prosto uporablja, distribuira in/ali predeluje pod njenimi pogoji.
Podrobnosti licence so dostopne na spletni strani \url{http://www.gnu.org/licenses/}.
}

\begin{center} 
\ \\ \vfill
{\em
Besedilo je oblikovano z urejevalnikom besedil \LaTeX. \\ Slike so izdelane s pomočjo jezika \pgfname/\tikzname. \\ Grafi so narisani
s pomočjo programa \gnuplot.}
\end{center}

\newpage

\ \thispagestyle{empty}

\newpage

%********************************************

% stran 3 med uvodnimi listi
\thispagestyle{empty}

Namesto te strani {\bf vstavite} original izdane teme diplomskega dela s podpisom mentorja in dekana ter \v zigom fakultete, ki ga diplomant
dvigne v študent\-skem referatu,  preden odda izdelek v vezavo!

\newpage

%********************************************

% stran 4 med uvodnimi listi je prazna 
\ \thispagestyle{empty}

\newpage

%********************************************

% stran 5 med uvodnimi listi

\thispagestyle{empty}

\vspace{1cm}
\begin{center} 
{\Large \textbf{IZJAVA O AVTORSTVU}}
\end{center}

\begin{center} 
{\Large diplomskega dela}
\end{center}

\vspace{1cm}
Spodaj podpisani/-a \hspace{0.5cm} Ime Priimek,

\vspace{0.5cm}
z vpisno številko \hspace{0.5cm} xxxxxxx,

\vspace{1cm}
sem avtor/-ica diplomskega dela z naslovom:
   
\vspace{0.5cm}
Naslov diplomskega dela

\vspace{1.5cm}
S svojim podpisom zagotavljam, da:
\begin{itemize}
	\item sem diplomsko delo izdelal/-a samostojno pod mentorstvom 
	
	prof. [doc.] dr. Ime Priimek
	
	in somentorstvom 
	
	prof. [doc.] dr. Ime Priimek
	
	\item	so elektronska oblika diplomskega dela, naslov (slov., angl.), povzetek (slov., angl.) ter ključne besede (slov., 			angl.) identični s tiskano obliko diplomskega dela
	\item soglašam z javno objavo elektronske oblike diplomskega dela v zbirki ''Dela FRI''.
\end{itemize}

\vspace{1cm}
V Ljubljani, dne xx.xx.20xx \hspace{1cm} Podpis avtorja/-ice:

\newpage 

%********************************************

% stran 6 med uvodnimi listi je prazna pri dvostranskem tiskanju
\ \thispagestyle{empty}

\newpage
\ \thispagestyle{empty}

%********************************************

% stran 7 med uvodnimi listi

\chapter*{Zahvala}

\thispagestyle{empty}

Na tem mestu se diplomant zahvali vsem, ki so kakorkoli pripomogli k uspešni izvedbi diplomskega dela.
\\**Zahvala podjetju Digital School D.O.O, za pomoč pri analizi učenja otrok z uporabo umetne inteligence. **


\newpage

%********************************************

% stran 8 med uvodnimi listi je prazna pri dvostranskem tiskanju
\ \thispagestyle{empty}

\newpage

%********************************************

\renewcommand\thepage{} 
\tableofcontents 
\renewcommand\thepage{\arabic{page}}

\thispagestyle{empty}


%********************************************

\chapter*{Seznam uporabljenih kratic in simbolov}

\thispagestyle{empty}

Seznam uporabljenih kratic in simbolov, ki morajo biti enotni v celotnem delu, ne glede na označevanje v uporabljenih virih.

%\cleardoublepage

\clearpage{\pagestyle{empty}\cleardoublepage}

%********************************************
%zacno se glavni listi, ki so numerirani z arabskimi stevilkami

\setcounter{page}{1}
\pagenumbering{arabic}

\chapter*{Povzetek}
\addcontentsline{toc}{chapter}{Povzetek}

Povzetek naj posreduje bralcu kratko vsebino dela. Zajema naj namen dela, področje, na katerega se delo nanaša,
uporabljene metode, poglavitne rezultate dela, zaključke in priporočila. 
Povzetek naj ne obsega več kot eno stran, obi\v cajno ima le 200 do 300 besed. Napiše se povsem na koncu,
ko je že jasna vsebina vseh ostalih poglavij.

Ta dokument vsebuje navodilo za izdelavo diplomskega dela v obliki in strukturi, ki je v teh navodilih predpisan za
pisanje diplomskih nalog. Struktura dokumenta je namenjena obojestranskemu tiskanju, kjer se novo poglavje vedno za\v cne na lihi strani.
V dejanski diplomi poglavja in podpogla\-vja  obi\v cajno niso tako kratka kot v teh navodilih.

Za oblikovanje tega dokumenta je bil uporabljen sistem \LaTeX.
Ve\v c o \LaTeX-u lahko izve\v s na spletni strani \texttt{http://www.ctan.org/}.
Kandidati, ki bodo svoje diplomsko delo oblikovali s pomo\v cjo
\LaTeX-a, lahko izvorno kodo tega dokumenta neposredno uporabijo kot vzorec za pisanje svoje diplomske naloge.

\vspace{1.3cm}
\noindent
{\large \bf Ključne besede:}

\vspace{0.5cm}
\noindent
diploma, mentor, zagovor, podaljšanje, pisanje, struktura


\chapter*{Abstract}

\addcontentsline{toc}{chapter}{Abstract}

Povzetek naj bo napisan v angleškem jeziku.

\vspace{1.3cm}
\noindent
{\large \bf Key words:}

\vspace{0.5cm}
\noindent
Ključne besede v angleškem jeziku.


%********************************************

\chapter{Uvod}

Umetna inteligenca je postala nepogrešljivo orodje pri programiranju, saj omogoča hitrejše in učinkovitejše reševanje problemov. Orodja za generiranje kode s pomočjo umetne inteligence, kot so GitHub Copilot, Amazon CodeWhisperer in ChatGPT podjetja OpenAI, postajajo vse bolj razširjena in se uporabljajo tako med začetniki kot med izkušenimi programerji. Ta orodja omogočajo generiranje kode na podlagi pozivov v naravnem jeziku ali delnih vnosov kode, kar bistveno poenostavi proces programiranja.

Motivacija za to delo izhaja iz želje po razumevanju, kako ta orodja vplivajo na proces učenja programiranja, še posebej med otroki. V dobi, ko postaja poznavanje programiranja vse pomembnejše, je ključnega pomena raziskati, ali lahko umetna inteligenca pospeši proces učenja in izboljša razumevanje programerskih konceptov. Hkrati je pomembno preučiti tudi morebitne slabosti uporabe takšnih orodij, da bi lahko izoblikovali najboljše prakse za njihovo uporabo v izobraževanju.

V naslednjih poglavjih bomo pregledali obstoječo literaturo in raziskave na področju hitrega inženirstva in uporabe orodij UI v programiranju. Nato bomo opisali metodologijo raziskave, kjer bomo predstavili postopek zbiranja podatkov in analizo uspešnosti otrok pri reševanju Python vaj z in brez pomoči ChatGPT. V empiričnem delu bomo predstavili rezultate raziskave in jih podrobno analizirali. Na koncu bomo povzeli ključne ugotovitve, prednosti in slabosti uporabe AI orodij pri učenju programiranja ter podali priporočila za nadaljnje raziskave in prakso.



\chapter{Pregled področja}
Razvoj programske kode je proces, ki vključuje pretvorbo specifikacije problema v programsko implementirano rešitev. Zgodnja orodja za olajšanje postopka so uporabljala strojno učenje za podporo razvijalcev skozi vse faze razvoja programske opreme in so med drugimi vključevala pridobivanje programske specifikacije in napoved uporabnikovega vnosa. \textcite{zhang2003machine} \\
Z današnjimi napredki na področjih globokega učenja in procesiranja naravnega jezika so se razvili napredni modeli za pomoč pri kodiranju, ki omogočajo avtomatsko dopolnjevanje in reševanje problemov v kodi v realnem času.
Obstaja veliko poskusov samodejnega prevajanja specifikacij v računalniško kodo s pomočjo formalnih modelov za avtomatsko generiranje kode *citat* ali pa s pomočjo strojno naučenega procesiranja naravnega jezika. Arhiterkure globokega učenja, ki so se dobro prilegajo procesiranju naravnega jezika so omogočile razvoj modelom ko sta GPT-2 in GPT-3 *citat*. Ti modeli lahko izvajajo
jezikovne naloge, kot sta prevajanje in odgovarjanje na vprašanja iz nabora podatkov CoQA. *vir* Po prilagajanju glede na specializiran nabor podatkov lahko modeli opravljajo naloge, kot so dopolnjevanje kode in načrtovanje strojne opreme. *citat* Najsodobnejši modeli imajo na milijarde parametrov, ki jih je mogoče naučiti, in so učeni na več milijonih programskih repozitorijev. V tem delu se bomo osredotočili na modela chatGPT-3.5 in Copilot. *več o teh orodjih* \\
\section{Sorodna literatura}

\cite{app13095783}
Namen tega poglavja je pregledati obstoječo literaturo povezano z uporabo umetne inteligence (UI) s poudarkom na hitrem inženiringu in pisanju programske opreme s pomočjo UI. 
Rudolph et al. (2023) so raziskovali priložnosti in izzive uporabe ChatGPT v izobraževanju. Ugotovili so, da ChatGPT ponuja velike prednosti, kot so omogočanje personaliziranega učenja in zagotavljanje takojšnjih povratnih informacij, vendar prinaša tudi izzive, vključno z morebitno prekomerno odvisnostjo od UI in potrebo po robustni digitalni pismenosti med študenti.
Uporaba modelov, ki temeljijo na UI, kot je ChatGPT, v računalniškem programiranju je obširno raziskana. Ti modeli so uspešni pri nalogah, kot so popravilo kode, povzemanje, dokončanje, popravljanje, razvrščanje in generiranje kode. Na primer, AlphaCode, napreden jezikovni model, je pokazal znatne zmožnosti pri različnih programerskih nalogah, kar poudarja potencial ChatGPT pri pomoči pri izobraževanju programiranja.
Obstoječa literatura poudarja preoblikovalni potencial ChatGPT v izobraževalnih okoljih, zlasti pri izboljšanju akademske uspešnosti in podpiranju personaliziranega učenja. Vendar pa poudarja tudi potrebo po etičnih premislekih in pomembnosti kombiniranja UI orodij s tradicionalnimi učnimi metodami za obvladovanje njihovih omejitev. Z nadaljnjim razvojem področja UI v izobraževanju bodo nadaljnje raziskave ključne za optimizacijo uporabe ChatGPT in podobnih orodij v različnih izobraževalnih kontekstih

? Veliki jezikovni modeli (angleško LLM) so vrsta umetne inteligence, zasnovana za razumevanje in ustvarjanje besedila, ki je blizu ljedem. Treniranje temelji na vnaprej naučenih vzorcih z obsežnim usposabljanjem na veliki količini podatkov, kot so spletne strani, knjige, članki. S pomočjo podatkov lahko model prepozna statistične vzorce in strukture človeškega jezika, kar omogoča, da se nauči slovnice, sintakse in semantične odnose v jeziku. Po končanem učenju se model navadno usmeri in prilagodi za določeno domeno ali nalogo, kot je na primer prevajanje, generiranje besedila, prepoznavanje slik, ipd. *cite* ?
\section{Asistenti na področju umetne inteligence} 

\section{Hitro programiranje}

\section{Programiranje s ChatGPT}
\cite{app13095783}
Napredek na področju umetne inteligence (UI), zlasti velikih jezikovnih modelov (VJM), je močno vplival na številna področja, vključno z izobraževanjem in razvojem programske opreme. Med temi inovacijami je ChatGPT, ki ga je razvilo podjetje OpenAI, postal pomembno orodje tako v izobraževalnih okoljih kot v praktičnih programerskih aplikacijah. To poglavje obravnava vlogo ChatGPT v hitrem inženiringu in razvoju programske opreme, ocenjuje njegove prednosti, izzive in prihodnje vplive.
Hiter inženiring vključuje oblikovanje vhodov za modele UI, da bi dosegli želene odzive. Kot model UI, usposobljen na obsežnih podatkovnih zbirkah, je ChatGPT pokazal izjemno sposobnost interpretacije in generiranja besedil, podobnih človeškim, na podlagi različnih pozivov. Ta sposobnost je neprecenljiva v izobraževalnih okoljih, kjer lahko pomaga pri ustvarjanju interaktivnih učnih okolij, zagotavljanju takojšnjih povratnih informacij in generiranju učne vsebine. Na primer, ChatGPT lahko pomaga študentom pri razlagi zapletenih konceptov na enostavnejši način, s čimer izboljša njihovo učno izkušnjo.
Na področju razvoja programske opreme ChatGPT nudi veliko podporo pri generiranju kode, odpravljanju napak in optimizaciji. Orodja, kot so GitHub Copilot in Amazon CodeWhisperer, ki jih poganjajo modeli UI, podobni ChatGPT, pomagajo razvijalcem pri generiranju delov kode na podlagi opisov v naravnem jeziku. Ta funkcionalnost ne samo pospešuje proces kodiranja, temveč tudi pomaga začetnikom pri razumevanju strukture in sintakse kode.
Integracija ChatGPT v izobraževalne sisteme predstavlja številne priložnosti in izzive. Pozitivno je, da ChatGPT lahko demokratizira dostop do izobraževanja, saj zagotavlja prilagojeno mentorstvo in pomoč študentom po vsem svetu. Omogoča interaktivne učne izkušnje in pomaga pri razumevanju zapletenih predmetov, kot je programiranje. Raziskave so pokazale, da študenti, ki uporabljajo ChatGPT za programerske naloge in reševanje problemov, dosegajo boljše rezultate in razumevanje.

Vendar pa uporaba ChatGPT prinaša tudi tveganja, vključno z morebitnim prekomernim zanašanjem na UI za reševanje problemov, kar lahko ovira razvoj kritičnega mišljenja. Poleg tega je potrebno nenehno preverjanje natančnosti in zanesljivosti vsebine, ki jo generira UI, da bi preprečili širjenje napačnih informacij.

\section{Programiranje z GithubCopilot}
Podjetje OpenAI je leta 2021 prvič predstavilo svoj nov produkt, OpenAI Codex, AI sistem na podlagi globokeda učenja, ki je bil naučen na več kot 50 milijonih GitHub repozitorijih. S podanim opisom programskega problema v naravnem jeziku Codex generira kodo za rešitev kot izhod. Zna razložiti kodo, jo prevesti med programskimi jeziki in poiskati ter popraviti napake v kodi.
Copilot temelji na družini modelov OpenAI Codex *citat*. Modeli Codex kot osnovo vzamejo model GPT-3 *citat*, ki ga nato prilagodijo na podlagi kode iz GitHuba. ?Njegov način tokenizacije je skoraj identičen z GPT-3, uporablja kodiranje parov bajtov za pretvorbo izvornega besedila v zaporedje žetonov, vendar je bil besedni zaklad GPT-3 razširjen z dodajanem namenskih žetonov za bele prostore (tj. žeton za dva presledka, žeton za
tri presledke, do 25 presledkov). To tokenizerju omogoča učinkoviteje
kodiranje izvorne kode. (ki ima veliko belih presledkov) \\
Pomembna lastnost, ki sta jo Codex in Copilot podedovala od
GPT-3, je, da ob pozivu ustvarita najverjetnejši odgovor
za poziv na podlagi tega, kar je bilo pridobljeno
iz izkušenj, pridobljenih pri usposabljanju. V kontekstu generiranja kode to pomeni,
da model ne bo nujno ustvaril najboljše kode (po
po kateri koli izbrani metriki - zmogljivost, varnost itd.),
temveč tisto, ki se najbolje ujema s predhodno in naučeno kodo.
Posledično na kakovost generirane kode močno vlivajo semantično nepomembne lastnosti poziva. \cite{9833571}

*vir = https://openai.com/index/openai-codex/
+the robots are coming vir


Preučevanje prednosti in slabosti, ocena pravilnosti in razlage rešitve, primerjava z ostalimi  AI asistenti.

\section{Ugotovitve in primerjava}

\chapter{Preučevanje hitrega programiranja na skupini otrok}
\section{Načrtovanje}

\section{Izpeljava}

\section{Ocena}

\chapter{Prijava diplomskega dela}





\chapter{Viri}
Za kakovostno diplomsko delo je pomembna uporaba vseh razpoložljivih doma\-čih in tujih strokovnih ter znanstvenih virov.

Yetiştiren, B., Özsoy, I., Ayerdem, M., \& Tüzün, E. (2023, April 21). Evaluating the Code Quality of AI-Assisted Code Generation Tools: An Empirical Study on GitHub Copilot, Amazon CodeWhisperer, and ChatGPT. Pridobljeno 28. 02. 2024 iz: https://arxiv.org/abs/2304.10778

Thorsten Brants, Ashok C. Popat, Peng Xu, Franz J. Och, and Jeffrey Dean. 2007. Large Language Models in Machine Translation. In Proceedings of the 2007 Joint Conference on Empirical Methods in Natural Language Processing and Computational Natural Language Learning (EMNLP-CoNLL), pages 858–867, Prague, Czech Republic. Association for Computational Linguistics.

\begin{thebibliography}{9}
\bibitem{yetistiren2023evaluating}
B.~Yetiştiren, I.~Özsoy, M.~Ayerdem, \& E.~Tüzün.
\newblock Evaluating the Code Quality of AI-Assisted Code Generation Tools: An Empirical Study on GitHub Copilot, Amazon CodeWhisperer, and ChatGPT.
\newblock Retrieved February 28, 2024, from \url{https://arxiv.org/abs/2304.10778} (April 21, 2023).

\bibitem{brants2007large}
Thorsten Brants, Ashok C. Popat, Peng Xu, Franz J. Och, \& Jeffrey Dean.
\newblock Large Language Models in Machine Translation.
\newblock In \textit{Proceedings of the 2007 Joint Conference on Empirical Methods in Natural Language Processing and Computational Natural Language Learning (EMNLP-CoNLL)}, pages 858--867, Prague, Czech Republic, 2007. Association for Computational Linguistics.

\bibitem{yong2023authors}
Yong Zheng.
\newblock Authors Info \& Claims.
\newblock In \textit{SIGITE '23: Proceedings of the 24th Annual Conference on Information Technology Education}, October 2023, pages 66--72, 2023. DOI: \url{10.1145/3585059.3611431}.
\bibitem{baidoo2023education}
D.~BAİDOO-ANU and L.~OWUSU ANSAH.
\newblock Education in the Era of Generative Artificial Intelligence (AI): Understanding the Potential Benefits of ChatGPT in Promoting Teaching and Learning.
\newblock \textit{Journal of AI}, 7(1), 52--62, 2023. DOI: \url{10.61969/jai.1337500}.

\end{thebibliography}



\end{document}
