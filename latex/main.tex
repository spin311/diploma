\documentclass[12pt,a4paper]{book}

% Uporabljeni paketi
\usepackage[utf8]{inputenc}
\usepackage{cmap}
\usepackage{float}
\usepackage{type1ec}
\usepackage[T1]{fontenc}
\usepackage{fancyhdr}
\usepackage{graphicx,epsfig}
\usepackage[slovene]{babel}
\usepackage{tikz}
\usepackage{csquotes}
\usepackage[
  backend=biber,
  style=numeric,
  sorting=none,
]{biblatex}
\usepackage{hyperref}
\addbibresource{literatura.bib}
\usepackage{longtable}
\usepackage{geometry}
\usepackage{array}
\geometry{a4paper, margin=1in}

% Nastavitev glave in repa strani
\pagestyle{fancy}
\fancyhead{}
\renewcommand{\chaptermark}[1]{\markboth{\textsf{Poglavje \thechapter:\ #1}}{}}
\renewcommand{\sectionmark}[1]{\markright{\textsf{\thesection\  #1}}{}}
\fancyhead[RE]{\leftmark}
\fancyhead[LO]{\rightmark}
\fancyhead[LE,RO]{\thepage}
\fancyfoot{}
\renewcommand{\headrulewidth}{0.0pt}
\renewcommand{\footrulewidth}{0.0pt}

\newcommand{\gnuplot}{\textbf{gnuplot}}
\newcommand{\pgfname}{\textsc{pgf}}
\newcommand{\tikzname}{Ti\emph{k}Z}

\begin{document}
\thispagestyle{empty}
\begin{center}
{\large 
UNIVERZA V LJUBLJANI\\
FAKULTETA ZA RAČUNALNIŠTVO IN INFORMATIKO\\
}

\vspace{3cm}
{\LARGE Svit Spindler}\\

\vspace{2cm}
\textsc{\textbf{\LARGE
Hiter inženiring pri razvoju programske opreme\\ 
}}

\vspace{2cm}
{ DIPLOMSKO DELO}\\
{ NA UNIVERZITETNEM ŠTUDIJU\\
}

\vspace{2cm} 
{\Large Mentor: izr. prof. dr. Dejan Lavbič}

\vfill
{\Large Ljubljana, 2024}
\end{center}

\newpage

%********************************************

% stran 3 med uvodnimi listi
\thispagestyle{empty}

Namesto te strani {\bf vstavite} original izdane teme diplomskega dela s podpisom mentorja in dekana ter \v zigom fakultete, ki ga diplomant
dvigne v študent\-skem referatu,  preden odda izdelek v vezavo!

\newpage

\newpage

\ \thispagestyle{empty}

\newpage

%********************************************

% stran 2 med uvodnimi listi
\thispagestyle{empty}

\vspace*{5cm}
{\small \noindent
To diplomsko delo je ponujeno pod licenco \textit{Creative Commons Priznanje avtorstva-Deljenje pod enakimi pogoji 2.5 Slovenija}
ali (po želji) novejšo različico.
To pomeni, da se tako besedilo, slike, grafi in druge sestavine dela kot tudi rezultati diplomskega dela lahko prosto distribuirajo,
reproducirajo, uporabljajo, dajejo v najem, priobčujejo javnosti in predelujejo, pod pogojem, da se jasno in vidno navede avtorja in naslov tega
dela in da se v primeru spremembe, preoblikovanja ali uporabe tega dela v svojem delu, lahko distribuira predelava le pod
licenco, ki je enaka tej.
Podrobnosti licence so dostopne na spletni strani \url{http://creativecommons.si/} ali na Inštitutu za
intelektualno lastnino, Streliška 1, 1000 Ljubljana.

\begin{center}% 0.66 / 0.89 = 0.741573033707865

\end{center}
}

\vspace*{1.5cm}
{\small \noindent
Izvorna koda diplomskega dela, njenih rezultatov in v ta namen razvite programske opreme je ponujena pod GNU General Public License,
različica 3 ali (po želji) novejšo različico. To pomeni, da se lahko prosto uporablja, distribuira in/ali predeluje pod njenimi pogoji.
Podrobnosti licence so dostopne na spletni strani \url{http://www.gnu.org/licenses/}.
}

\begin{center} 
\ \\ \vfill
{\em
Besedilo je oblikovano z urejevalnikom besedil \LaTeX. \\ Slike so izdelane s pomočjo jezika \pgfname/\tikzname. \\ Grafi so narisani
s pomočjo programa \gnuplot.}
\end{center}

\newpage

\ \thispagestyle{empty}

\newpage

%********************************************

% stran 3 med uvodnimi listi
\thispagestyle{empty}

Namesto te strani {\bf vstavite} original izdane teme diplomskega dela s podpisom mentorja in dekana ter \v zigom fakultete, ki ga diplomant
dvigne v študent\-skem referatu,  preden odda izdelek v vezavo!

\newpage

%********************************************

% stran 4 med uvodnimi listi je prazna 
\ \thispagestyle{empty}

\newpage

%********************************************

% stran 5 med uvodnimi listi

\thispagestyle{empty}

\vspace{1cm}
\begin{center} 
{\Large \textbf{IZJAVA O AVTORSTVU}}
\end{center}

\begin{center} 
{\Large diplomskega dela}
\end{center}

\vspace{1cm}
Spodaj podpisani/-a \hspace{0.5cm} Ime Priimek,

\vspace{0.5cm}
z vpisno številko \hspace{0.5cm} xxxxxxx,

\vspace{1cm}
sem avtor/-ica diplomskega dela z naslovom:
   
\vspace{0.5cm}
Naslov diplomskega dela

\vspace{1.5cm}
S svojim podpisom zagotavljam, da:
\begin{itemize}
    \item sem diplomsko delo izdelal/-a samostojno pod mentorstvom prof. [doc.] dr. Ime Priimek in somentorstvom prof. [doc.] dr. Ime Priimek
    \item so elektronska oblika diplomskega dela, naslov (slov., angl.), povzetek (slov., angl.) ter ključne besede (slov., angl.) identični s tiskano obliko diplomskega dela
    \item soglašam z javno objavo elektronske oblike diplomskega dela v zbirki ''Dela FRI''.
\end{itemize}

\vspace{1cm}
V Ljubljani, dne xx.xx.20xx \hspace{1cm} Podpis avtorja/-ice:

\newpage 

%********************************************

% stran 6 med uvodnimi listi je prazna pri dvostranskem tiskanju
\ \thispagestyle{empty}

\newpage
\ \thispagestyle{empty}

%********************************************

% stran 7 med uvodnimi listi

\chapter*{Zahvala}

\thispagestyle{empty}

Na tem mestu se diplomant zahvali vsem, ki so kakorkoli pripomogli k uspešni izvedbi diplomskega dela.
\\**Zahvala podjetju Digital School D.O.O, za pomoč pri analizi učenja otrok z uporabo umetne inteligence. **


\newpage

%********************************************

% stran 8 med uvodnimi listi je prazna pri dvostranskem tiskanju
\ \thispagestyle{empty}

\newpage

%********************************************

\renewcommand\thepage{} 
\tableofcontents 
\renewcommand\thepage{\arabic{page}}

\thispagestyle{empty}


%********************************************

\chapter*{Seznam uporabljenih kratic in simbolov}

\thispagestyle{empty}

Seznam uporabljenih kratic in simbolov, ki morajo biti enotni v celotnem delu, ne glede na označevanje v uporabljenih virih.

\begin{table}[h!]
    \centering
    \begin{tabular}{|l|l|}
        \hline
        UI & umetna inteligenca \\
        \hline
        PNJ & procesiranje naravnih jezikov \\
        \hline
        GCC & GitHub Copilot Chat \\
        \hline
        IDE & integrirano razvojno okolje \\
        \hline
    \end{tabular}
    \caption{Seznam kratic in njihovih pomenov}
    \label{tab:abbreviations}
\end{table}

%\cleardoublepage

\clearpage{\pagestyle{empty}\cleardoublepage}

%********************************************
%zacno se glavni listi, ki so numerirani z arabskimi stevilkami

\setcounter{page}{1}
\pagenumbering{arabic}

\chapter*{Povzetek}
\addcontentsline{toc}{chapter}{Povzetek}

Povzetek naj posreduje bralcu kratko vsebino dela. Zajema naj namen dela, področje, na katerega se delo nanaša,
uporabljene metode, poglavitne rezultate dela, zaključke in priporočila. 
Povzetek naj ne obsega več kot eno stran, obi\v cajno ima le 200 do 300 besed. Napiše se povsem na koncu,
ko je že jasna vsebina vseh ostalih poglavij.

Ta dokument vsebuje navodilo za izdelavo diplomskega dela v obliki in strukturi, ki je v teh navodilih predpisan za
pisanje diplomskih nalog. Struktura dokumenta je namenjena obojestranskemu tiskanju, kjer se novo poglavje vedno za\v cne na lihi strani.
V dejanski diplomi poglavja in podpogla\-vja  obi\v cajno niso tako kratka kot v teh navodilih.

Za oblikovanje tega dokumenta je bil uporabljen sistem \LaTeX.
Ve\v c o \LaTeX-u lahko izve\v s na spletni strani \texttt{http://www.ctan.org/}.
Kandidati, ki bodo svoje diplomsko delo oblikovali s pomo\v cjo
\LaTeX-a, lahko izvorno kodo tega dokumenta neposredno uporabijo kot vzorec za pisanje svoje diplomske naloge.

\vspace{1.3cm}
\noindent
{\large \bf Ključne besede:}

\vspace{0.5cm}
\noindent
diploma, mentor, zagovor, podaljšanje, pisanje, struktura


\chapter*{Abstract}

\addcontentsline{toc}{chapter}{Abstract}

Povzetek naj bo napisan v angleškem jeziku.

\vspace{1.3cm}
\noindent
{\large \bf Key words:}

\vspace{0.5cm}
\noindent
Ključne besede v angleškem jeziku.


%********************************************

\chapter{Uvod}

Orodja za dopolnjevanje in generiranje kode so pri programiranju postala zelo popularna, saj omogočajo hitrejše in učinkovitejše reševanje problemov. Orodja za generiranje kode s pomočjo umetne inteligence, kot so GitHub Copilot, Amazon CodeWhisperer in ChatGPT podjetja OpenAI se uporabljajo tako med začetniki kot tudi med izkušenimi programerji. Ta orodja omogočajo generiranje kode na podlagi pozivov v naravnem jeziku ali delnih vnosov kode, kar bistveno poenostavi proces programiranja.

Namen te študije je preučiti prednosti in slabosti najpopularnejših orodij za umetno inteligenco (UI) ter raziskati, kako omenjena orodja vplivajo na proces učenja in programiranja pri začetnih programerjih s preučevanjem hitrosti in kvalitete kode na različnih programerskih problemih v jeziku Python. V dobi, ko postaja UI vedno bolj dostopna in popularna, je ključnega pomena raziskati, ali lahko UI pospeši proces učenja in izboljša razumevanje programerskih konceptov.  

V naslednjih poglavjih bomo pregledali razvoj programske kode in vlogo strojnega učenja ter naprednih jezikovnih modelov pri avtomatizaciji tega procesa. Najprej bomo raziskali zgodovinski razvoj orodij za podporo razvijalcem in zgodnje uporabe strojnega učenja pri pridobivanju specifikacij in napovedovanju uporabniškega vnosa. Nato bomo obravnavali sodobne arhitekture globokega učenja, kot sta GPT-3 in GPT-4, ki omogočajo napredne funkcionalnosti avtomatskega dopolnjevanja in generiranja kode.



\chapter{Pregled področja}
Narava in kompleksnost programske opreme sta se v zadnjih 30 letih močno spremenili. V sedemdesetih letih 20. stoletja so aplikacije delovale na enem samem procesorju, proizvajale alfanumerične izhode in prejemale svoj vhod iz linearnega vira. Današnje aplikacije so veliko bolj zapletene; običajno imajo grafični uporabniški vmesnik in uporabljajo arhitekturo odjemalec-strežnik. Pogosto delujejo na dveh ali več procesorjih, pod različnimi operacijskimi sistemi in na geografsko porazdeljenih strojih. Redko katero področje v zgodovini se je  razvilo tako hitro kot razvoj programske opreme. Zato je nujno potrebno izboljševati koncepte, strategije in prakse programskega inženiringa, da bi se izognili konfliktom in izboljšali proces razvoja programske opreme, da lahko pravočasno in v okviru proračuna zagotovimo kakovostno programsko opremo, ki jo je mogoče vzdrževati.
V naslednjih poglavjih se bomo posvetili pregledu programskega inženirstva in njegove zgodovine ter sodobnega inženirstva pozivov, ki poskuša ta proces izboljšati in pospešiti.

\cite{aggarwal2005software}
\section{Programsko inženirstvo}
V dobi digitalizacije je programska oprema postala ključna za napredek na skoraj vseh področjih človeškega delovanja. Programiranje samo ne zadostuje več za izdelavo velikih programov. Obstajajo resne težave pri kakovosti,stroških, pravočasnosti in vzdrževanju številnih izdelkov programske opreme.
Cilj programskega inženiringa je reševanje teh težav z izdelavo kakovostne programske opreme, ki je narejena pravočasno in v okviru proračuna. Da bi dosegli ta cilj, se moramo osredotočiti tako na kakovost izdelka kot tudi na procese, uporabljene za razvoj izdelka.
Na prvi konferenci o programskem inženiringu leta 1968 je Fritz Bauer definiral programski inženiring kot "Vzpostavitev in uporaba dobrih inženirskih principov za pridobitev ekonomično razvite programske opreme, ki je zanesljiva in učinkovito deluje na resničnih strojih". Stephen Schach je enak pojem opredelil kot "Področje, katerega cilj je izdelava kakovostne programske opreme, ki je izdelana pravočasno, v okviru predvidenega proračuna in izpolnjuje zastavljene zahteve.". Obe definiciji sta priljubljeni in sprejemljivi za večino. Vendar pa se zaradi povečanja stroškov vzdrževanja programske opreme cilj zdaj premika k izdelavi kakovostne programske opreme, ki jo je mogoče vzdrževati, je dostavljena pravočasno, v okviru proračuna in prav tako izpolnjuje njene zahteve.
Kriza programske opreme nas spremlja že od leta 1970. Od takrat se je računalniška industrija z računalniško revolucijo in nedavno z revolucijo omrežja, ki jo je sprožila in/ali pospešila eksplozivna širitev interneta in v zadnjem času spleta, razvijala z bliskovito hitrostjo.
Računalniška industrija je zagotavljala eksponentno izboljšanje razmerja med ceno in zmogljivostjo, vendar se težave s programsko opremo niso zmanjšale. Programska oprema še vedno zamuja, presega proračun in je polna napak. Po zadnjem IBM-ovem poročilu je "31 \% projektov preklicanih, še preden so končani, 53 \% projektov presega oceno stroškov za povprečno 189 \%, na vsakih 100 projektov pa je 94 ponovnih zagonov".

Proces razvoja programske opreme je postopek, s katerim ustvarjamo programsko opremo. Ta se razlikuje od organizacije do organizacije. Za preživetje v vse bolj konkurenčni dobi je za razvoj programske opreme potrebno več kot le zaposliti pametne in vešče razvijalce ter kupiti najnovejša razvojna orodja. Uporabljati moramo tudi učinkovite procese razvoja programske opreme, da lahko razvijalci sistematično uporabljajo najboljše tehnične in vodstvene prakse za uspešno dokončanje svojih projektov. Številne organizacije, ki se ukvarjajo s programsko opremo, si prizadevajo za izboljšanje procesov razvoja programske opreme kot načina za izboljšanje kakovosti, produktivnosti in predvidljivosti razvoja programske opreme ter vzdrževanja.
Zdi se preprosto in v literaturi je mogoče najti številne zgodbe o uspehu podjetij, ki so bistveno izboljšala svoje zmogljivosti za razvoj programske opreme in vodenje projektov. Vendar mnogim organizacijam ne uspe doseči občutnih in trajnih izboljšav v načinu vodenja svojih projektov. V nadaljevanju obravnavamo nekaj razlogov, zakaj je težko izboljšati proces programske opreme.

1. Premalo časa. Zaradi nerealnih časovnih rokov ni dovolj časa za opravljanje bistvenega projektnega dela. Nobena programerska skupina nima prostega časa na pretek, ki bi ga lahko namenila raziskovanju, kaj je narobe z njihovimi trenutnimi razvojnimi procesi in kaj bi morali delati drugače. Stranke in nadrejeni zahtevajo čim več programske opreme višje kakovosti v čim krajšem času. Zato vedno primanjkuje časa. Ena od posledic tega je, da programske organizacije morda pravočasno dostavijo izdajo 1.0, vendar morajo skoraj takoj zatem dostaviti različico 1.01, da odpravijo nedavno odkrite napake.

2. Pomanjkanje znanja. Druga ovira za izboljšanje procesov je, da se zdi, da mnogi razvijalci programske opreme ne poznajo najboljših industrijskih praks. Običajno razvijalci programske opreme ne porabijo veliko časa za branje literature, da bi se seznanili z najboljšimi znanimi načini razvoja programske opreme. Razvijalci lahko kupijo knjige o Javi, Visual Basicu ali ORACLE, vendar na svojih knjižnih policah ne iščejo ničesar o procesih, testiranju ali kakovosti.
Zavedanje industrije o okvirih za izboljšanje procesov, kot sta model zrelosti zmogljivosti in ISO 9001 standard za programsko opremo, se je v zadnjih letih povečalo, vendar učinkovita in smiselna uporaba še vedno ni tako pogosta. Številne priznane najboljše prakse, ki so na voljo v literaturi, se v svetu razvoja programske opreme preprosto ne uporabljajo pogosto.

Cilj inženirstva programske opreme je zagotoviti modele in postopke, ki vodijo k izdelavi dobro dokumentirane programske opreme, ki jo je mogoče vzdrževati na predvidljiv način. Pri zrelem procesu bi moralo biti mogoče vnaprej določiti, koliko časa in truda bo potrebnega za izdelavo končnega izdelka. To lahko storimo le s pomočjo podatkov iz preteklih izkušenj, kar zahteva, da moramo proces programske opreme ustrezno poznati in oceniti.
Organizacije, ki se ukvarjajo z razvojem programske opreme, pri razvoju programskega izdelka sledijo določenemu procesu. V nezrelih organizacijah proces običajno ni zapisan. V zrelih organizacijah je proces zapisan in se aktivno nadzoruje. Ključna enota vsakega procesa razvoja programske opreme je model življenjskega cikla, na katerem proces temelji. Izbran model življenjskega cikla lahko bistveno vpliva na skupne stroške življenjskega cikla, povezane s programskim izdelkom. Življenjski cikel programske opreme se začne z raziskovanjem zamisli in konča z umikom programske opreme iz uporabe.
V standardu IEEE terminologije programskega inženirstva je življenjski cikel programske opreme naslednji:
"Časovno obdobje, ki se začne, ko je programski izdelek zasnovan, in konča, ko izdelek ni več na voljo za uporabo. Življenjski cikel programske opreme običajno vključuje fazo zahtev, fazo načrtovanja, fazo izvajanja, fazo preizkušanja, fazo namestitve in preverjanja, fazo delovanja in vzdrževanja ter včasih fazo umika".
Model življenjskega cikla programske opreme je posebna abstrakcija, ki predstavlja življenjski cikel programske opreme. Model življenjskega cikla programske opreme se pogosto imenuje življenjski cikel razvoja programske opreme (SDLC).
\subsection{MODELI SDLC}

Predlagani so bili različni modeli življenjskega cikla, ki temeljijo na nalogah, vključenih v razvoj in vzdrževanje programske opreme. V tem poglavju je obravnavanih nekaj znanih modelov življenjskih ciklov.
2.1.1 Model gradnje in popravljanja
Včasih je izdelek izdelan brez specifikacij ali kakršnega koli poskusa načrtovanja. Namesto tega razvijalec preprosto zgradi izdelek, ki ga predela tolikokrat, kolikor je potrebno, da zadovolji naročnika.
To je ad hoc pristop in ni dobro opredeljen. V osnovi gre za preprost dvofazni model. Prva faza je pisanje kode, naslednja faza pa njeno popravljanje. Popravljanje v tem kontekstu je lahko popravljanje napak ali dodajanje dodatnih funkcionalnosti.
Čeprav se ta pristop lahko dobro obnese pri majhnih programskih nalogah, dolgih 100 ali 200 vrstic, je ta model popolnoma nezadovoljiv za programsko opremo kakršne koli večje obsežnosti. Koda kmalu postane nepopravljiva in neizboljšljiva. Ni prostora za strukturirano ali podrobno načrtovanje ali kateri koli vidik razvojnega procesa. Stroški razvoja s tem pristopom so dejansko zelo visoki v primerjavi s stroški pravilno določenega in skrbno zasnovanega izdelka. Poleg tega je lahko vzdrževanje izdelka brez specifikacije ali projektne dokumentacije zelo težavno.

2.1.2 Slapovni model

Najbolj znan model je slapovni model, ki je sestavljen iz petih faz: analiza in specifikacija zahtev, načrtovanje, izvedba in testiranje enot, integracija in sistemsko testiranje ter delovanje in vzdrževanje. 
Faze si vedno sledijo v tem vrstnem redu in se ne prekrivajo. Razvijalec mora zaključiti vsako fazo, preden začne z naslednjo fazo. Model se imenuje "Slapovni model", ker njegov diagramski prikaz spominja na kaskado slapov.

\begin{figure}[H]
    \centering
    \includegraphics[width=1\linewidth]{slapovni-model.jpg}
    \caption{Faze slapovnega modela}
    \label{fig:enter-label}
\end{figure}

1. Faza analize in specifikacije zahtev.

Cilj te faze je razumeti natančne zahteve naročnika in jih ustrezno dokumentirati. Ta dejavnost se običajno izvaja skupaj z naročnikom, saj je cilj dokumentirati vse zahteve glede funkcij, delovanja in vmesnikov za programsko opremo. Zahteve opisujejo „kaj“ sistema in ne „kako“. V tej fazi nastane obsežen dokument, napisan v naravnem jeziku, ki vsebuje opis, kaj bo sistem počel, ne da bi opisoval, kako bo to storjeno. Nastali dokument je znan kot dokument specifikacije zahtev za programsko opremo (angl. SRS). Dokument SRS lahko deluje kot pogodba med razvijalcem in stranko. Če razvijalec ne izvede celotnega sklopa zahtev, lahko to pomeni, da ni izvedel sistema, za katerega je bila sklenjena pogodba.


2. Faza načrtovanja.

 V prejšnji fazi se izdela dokument SRS, ki vsebuje natančne zahteve naročnika. Cilj faze načrtovanja je pretvoriti specifikacijo zahtev v strukturo, ki je primerna za implementacijo v nekem programskem jeziku. Tu se opredeli splošna arhitektura programske opreme ter izvede delo na visoki ravni in podrobno definiranje. To delo se dokumentira in je znano kot dokument SDD (software design description). Informacije v dokumentu SDD morajo zadostovati za začetek faze pisanja kode.
3. Faza izvajanja in testiranja enote.

V tej fazi se izvede načrtovanje. Če je SDD celovit, faza izvajanja ali kodiranja poteka nemoteno, saj so vse informacije, ki jih potrebujejo razvijalci programske opreme, vsebovane v SDD.
Med testiranjem so glavne dejavnosti osredotočene na pregledovanje in spreminjanje kode. Sprva se majhni moduli preizkušajo ločeno od preostalega programskega izdelka. S testiranjem modula v izolaciji so povezane določene težave. Kako zagnati modul, ne da bi ga karkoli poklicalo, ter kako ga poklicati ali morda izpisati vmesne vrednosti, pridobljene med izvajanjem? Takšne težave se rešijo v tej fazi, moduli pa se testirajo po tem, ko se napiše nekaj splošne kode.


4. Faza integracije in testiranja sistema: 
 To je zelo pomembna faza. Učinkovito testiranje bo prispevalo k zagotavljanju kakovostnejših programskih izdelkov, bolj zadovoljnih uporabnikov, nižjih stroškov vzdrževanja ter natančnejših in zanesljivejših rezultatov. Je zelo draga dejavnost in lahko porabi od tretjine do polovice stroškov tipičnega razvojnega projekta.
Kot vemo, je namen testiranja enot ugotoviti, ali je vsak neodvisni modul pravilno implementiran. To daje malo možnosti za ugotavljanje, ali so pravilni tudi vmesniki med moduli, zato se izvaja integracijsko testiranje. Sistemsko testiranje vključuje testiranje celotnega sistema, medtem ko je programska oprema del sistema. To je bistvenega pomena za vzpostavitev zaupanja v razvijalce, preden je programska oprema dostavljena stranki ali ponovno dana v najem na trgu.


5. Faza delovanja in vzdrževanja. Vzdrževanje programske opreme je naloga, s katero se mora soočiti vsaka razvojna skupina, ko je programska oprema dostavljena stranki, nameščena in deluje. Zato se z izdajo programske opreme začne faza delovanja in vzdrževanja v življenjskem ciklu.
Poraba časa in prizadevanja, potrebna za vzdrževanje delovanja programske opreme po izdaji, sta zelo pomembna. Kljub temu, da gre za zelo pomembno in zahtevno nalogo, je to običajno slabo obvladan preglavica, s katero se nihče ne želi soočiti.
Vzdrževanje programske opreme je zelo obsežna dejavnost, ki vključuje odpravljanje napak, izboljšanje zmogljivosti, odstranjevanje zastarelih zmogljivosti in optimizacijo. Namen te faze je ohraniti kakovost programske opreme v daljšem časovnem obdobju. Ta faza lahko traja od 5 do 50 let, medtem ko razvoj lahko traja od 1 do 3 let.
Ta model je enostaven za razumevanje in krepi koncept „definiraj pred načrtovanjem“ in „načrtuj pred kodo“. Ta model pričakuje popolne in natančne zahteve že na začetku procesa, kar je nerealno. Delujoča programska oprema je na voljo šele razmeroma pozno v procesu, kar odloži odkrivanje resnih napak. Prav tako ne vključuje nobene vrste ocene tveganja.

Težave slapovnega modela
\begin{enumerate}
    \item Na začetku projekta je težko opredeliti vse zahteve.
    \item Model ni primeren za prilagajanje kakršnim koli spremembam.
    \item Delujoča različica sistema je na voljo šele v poznih fazah projekta.
    \item Model ni primeren za velike projekte.
    \item Resnični projekti so redko zaporedni.
\end{enumerate}
Zaradi teh slabosti je treba uporabo modela slapu omejiti na primere, ko so zahteve in njihovo izvajanje dobro poznani. Na primer, če ima organizacija izkušnje z razvojem računovodskih sistemov, potem bi lahko izgradnjo novega računovodskega sistema na podlagi obstoječih zasnov zlahka upravljala z modelom slapu.


2.1.3 Model izdelave prototipov

Pomanjkljivost modela slapu, kot je bilo obravnavano v zadnjem poglavju, je, da je delujoča programska oprema na voljo šele v poznih fazah procesa, kar upočasni odkrivanje pomembnih napak. Alternativa temu je, da se namesto razvoja dejanske programske opreme najprej razvije delujoči prototip programske opreme. Delovni prototip se razvije v skladu s trenutno razpoložljivimi zahtevami. V osnovi ima omejene funkcionalne zmogljivosti, nizko zanesljivost in nepreverjeno zmogljivost (običajno nizko).
Razvijalci ta prototip uporabijo za izpopolnitev zahtev in pripravo končnega specifikacijskega dokumenta. Ker je delovni prototip ocenila stranka, je upravičeno pričakovati, da bo nastali specifikacijski dokument pravilen. Ko je prototip izdelan, ga pregleda stranka. Običajno ta pregled daje povratne informacije razvijalcem, ki pomagajo odpraviti negotovosti v zahtevah programske opreme, in začne iteracijo izpopolnjevanja, da bi še bolj razjasnili zahteve.
Prototip je lahko uporaben program, vendar ni primeren kot končni izdelek programske opreme. Razlog za to je lahko slabo delovanje, vzdrževanje ali splošna kakovost. Koda prototipa se zavrže, vendar izkušnje, pridobljene pri razvoju prototipa, pomagajo pri razvoju dejanskega sistema. Zato lahko razvoj prototipa povzroči dodatne stroške, vendar se lahko izkaže, da so skupni stroški nižji od stroškov enakovrednega sistema, razvitega po modelu slapu.

\cite{aggarwal2005software}

Obstaja veliko tradicionalnih metod in pristopov k razvoju programske opreme, kot je predhodno omenjeni slapovni pristop, ter iterativni in inkrementalni pristop, spiralni pristop, evolucijski pristop itd. Ti pristopi se včasih imenujejo načrtovani pristopi za razvoj programske opreme ali težki pristopi.  Ti pristopi so zelo uporabni, kadar je treba razviti obsežno kompleksno programsko opremo, saj lahko pomagajo odpraviti staromoden neformalen razvoj programske opreme in zagotoviti visokokakovostno programsko opremo na sistematičen način, ki izpolnjuje zahteve uporabnikov v vnaprej določenem omejenem času. težava tradicionalnih metod razvoja programske opreme je, da potrebujejo zelo zahtevne postopke, kot so: 
\begin{enumerate}
    \item Načrt projekta mora biti narejen vnaprej. 
    \item Potrebujemo zapisane zahteve za programsko opremo.
    \item Popolno načrtovanje, ki izpolnjuje zapisane zahteve.
    \item Izdelava programske kode, ki izpolnjuje vse zapisane zahteve in načrte.
    \item Popolno testiranje programske opreme in preverjanje, ali izpolnjuje zahteve in načrt.
\end{enumerate}
Številni projekti, ki sledijo tradicionalnim metodam razvoja programske opreme, se soočajo z velikimi težavami, zlasti pri vzdrževanju in spremembah na podlagi novih zahtev uporabnikov. Nekatere od teh zahtev lahko privedejo do velikih sprememb, ki se štejejo za velik problem pri razvoju programske opreme.  Zaradi vsega tega je potrebna lahka metoda razvoja programske opreme, katere glavni cilj je pospešiti razvoj in se učinkovito odzvati na spremembe, ki jih zahtevajo uporabniki.  Ta lahka metoda razvoja programske opreme se imenuje agilne metode razvoja programske opreme. 
\cite{AlSaqqa2020AgileSD}

„Agilno gibanje“ v industriji programske opreme je ugledalo luč sveta s projektom Agile Manifest razvoja programske opreme, ki ga je objavila skupina razvijalcev programske opreme praktikov in svetovalcev leta 2001 (Beck et al. 2001; Cockburn 2002a). Osrednje vrednote, ki so jih spoštovali agilisti, so predstavljene v nadaljevanju:
- posamezniki in interakcije pred procesi in orodji
- Delujoča programska oprema namesto izčrpne dokumentacije
- sodelovanje s strankami namesto pogajanj o pogodbah
- Odzivanje na spremembe namesto sledenja načrtu
Te osrednje vrednote, ki jih upošteva agilna skupnost, so:
Prvič, agilno gibanje poudarja odnos in skupnost
razvijalcev programske opreme in človeško vlogo, ki se odraža v pogodbah, v nasprotju z institucionaliziranim procesom in razvojnim orodjem. V obstoječih agilnih praksah, se to kaže v tesnih odnosih v skupini, tesnem delovnem okolju
in drugih postopkih, ki spodbujajo timski duh.
Drugič, ključni cilj ekipe za programsko opremo je nenehno izdelovanje preizkušene delujoče programske opreme. Nove izdaje se pripravljajo v pogostih časovnih intervalih, v nekaterih pristopih celo vsako uro ali vsak dan, vendar bolj običajno na dva meseca ali mesec. 
Razvijalci so pozvani, naj bo koda preprosta, enostavna in tehnično
čim bolj napredna, s čimer se zmanjša breme dokumentacije na ustrezno raven.
Tretjič, odnos in sodelovanje med razvijalci in strankami daje prednost pred strogimi pogodbami, čeprav pomembnost dobre pogodbe narašča z enako hitrostjo kot velikost projekta programske opreme.
Sam proces pogajanj je treba obravnavati kot sredstvo za doseganje in
ohranjanje uspešnega odnosa. S poslovnega vidika je agilni razvoj osredotočen na zagotavljanje poslovne vrednosti takoj na začetku projekta, s čimer se zmanjšajo tveganja neizpolnitve pogodbe.
Četrtič, razvojna skupina, ki jo sestavljajo tako razvijalci programske opreme kot predstavniki strank, mora biti dobro obveščena, usposobljena in pooblaščena za upoštevanje morebitnih potreb po prilagoditvah, ki se pojavijo med življenskim ciklom razvojnega procesa. To pomeni, da so udeleženci pripravljeni na spremembe in da so tudi obstoječe pogodbe oblikovane z orodji, ki podpirajo in omogočajo te
izboljšave.
Highsmith in Cockburn pravita, da „novost pri agilnih metodah niso prakse, ki jih uporabljajo, temveč njihovo priznavanje ljudi kot glavnih dejavnikov uspeha projekta, skupaj z intenzivnim osredotočanjem na učinkovitost in okretnost. 
\cite{DBLP:journals/corr/abs-1709-08439}

\section{Inženiring pozivov}

Umetna inteligenca (UI) je pokazala velik potencial pri prispevanju k avtomatizaciji izvajanja številnih zahtevnih nalog programskega inženirstva, kot so generiranje kode , popravljanje programov in povzemanje ter razlaga kode. Najsodobnejši pristopi pri generiranju kode uporabljajo modele umetne inteligence za samodejno izdelavo delnih ali celotnih programov na podlagi opisov v naravnem jeziku ali nekaterih vhodnih podatkov kode. Na področju popravljanja programov postajajo modeli, ki temeljijo na nevronskih mrežah, de facto standard za učenje generiranja popravkov kode, s čimer se zmanjša ročno odpravljanje napak ter poveča zanesljivost in varnost programske opreme. Generativni modeli umetne inteligence so ob vhodni kodi zmožni izdelati opise le te v naravnem jeziku in tako prispevajo k lažjemu razumevanju kode, vzdrževanju kode in njeni ponovni uporabi. Nedavni pojav obsežnih jezikovnih modelov (LLM) je bil v družbi na splošno deležen velikega zanimanja. Na področju inženirstva programske opreme je prinesel avtomatizacijo, ki jo poganja umetna inteligenca, na novo raven. Modeli LLM, ki so predhodno usposobljeni na velikih količinah izvorne kode in podatkov o naravnem jeziku, so pokazali izjemne sposobnosti pri razumevanju struktur kode in ustvarjanju kode ali besedil. Napredek, ki ga vodijo LLM, je še povečal učinkovitost samodejnih tehnik za različne izzive programskega inženirstva, kot so generiranje kode, popravljanje programov in povzemanje kode. Pomemben primer LLM je OpenAI-jev Codex. Codex je bil uspešno uporabljen za vrsto nalog inženiringa programske opreme, pri čemer je dokazal, da je sposoben generirati natančne in učinkovite odlomke kode, prepoznati in odpraviti napake in ranljivosti programske opreme ter zagotoviti povzetke za segmente kode. Eksperimentalni rezultati v literaturi kažejo na velik potencial LLM v skupnosti za razvoj programske opreme, čeprav skupna dosežena učinkovitost ostaja razmeroma omejena. ChatGPT, nedavno predstavljen LLM, je v skupnosti inženirjev programske opreme vzbudil veliko zanimanja zaradi njegovih zmogljivosti pri dolgoletnih sanjah inženirjev programske opreme: samodejnem popravljanju programske opreme z minimalnim človeškim posredovanjem. Rezultati kažejo, da bo ChatGPT spremenil to področje in da ima programsko inženirstvo, ki temelji na LLM, svetlo prihodnost. 
 \cite{tian2023chatgptultimateprogrammingassistant}

Hiter inženiring je razmeroma novo področje raziskav, ki se nanaša na prakso oblikovanja, izpopolnjevanja in izvajanja pozivov ali navodil, ki usmerjajo izhodne podatke LLM za pomoč pri različnih nalogah. Gre za prakso učinkovitega sodelovanja s sistemi umetne inteligence za optimizacijo njihovih koristi.
\cite{info:doi/10.2196/50638}

Razvoj programske kode je proces, ki vključuje pretvorbo specifikacije problema v programsko implementirano rešitev. Zgodnja orodja za olajšanje postopka so uporabljala strojno učenje za podporo razvijalcev skozi vse faze razvoja programske opreme in so med drugimi vključevala pridobivanje programske specifikacije in napoved uporabnikovega vnosa.
Med zgodnjo opremo sodijo orodja za dopolnjevanje kode, ki so na voljo v večini urejevalnikov kode in ki izpišejo kontekstualno relevantne spremenljivke, metode, razrede in druge dele kode v obliki plavajočega menija. Razvijalci se lahko z raziskovanjem in izbiranjem v tem meniju izognejo pogostim slovničnim in logičnim napakam, zmanjšajo število pritiskov na tipke in raziskujejo nove API-je namesto da porabljajo čas in napor za iskanje dokumentacije. Nekatera znana orodja za dopolnjevanje kode vključujejo IntelliSense v Visual Studio Code in vgrajeno dopolnjevanje kode v orodjih JetBrains IDE2. 
Čeprav lahko ta orodja izpisujejo odlomke kode, se bistveno razlikujejo od generatorjev kode.
Z današnjimi napredki na področjih globokega učenja in procesiranja naravnega jezika so se razvili napredni modeli za pomoč pri kodiranju, ki omogočajo avtomatsko dopolnjevanje in reševanje problemov v kodi v realnem času.
Obstaja veliko poskusov samodejnega prevajanja specifikacij v računalniško kodo s pomočjo formalnih modelov za avtomatsko generiranje kode ali pa s pomočjo strojno naučenega procesiranja naravnega jezika. Arhitekture globokega učenja, ki se dobro prilegajo procesiranju naravnega jezika, so omogočile razvoj modelom ko sta GPT-2 in GPT-3. Ti modeli lahko izvajajo jezikovne naloge, kot sta prevajanje in odgovarjanje na vprašanja iz nabora podatkov pogovornega naziva za odgovarjanje na vprašanja (angl. CoQA). Po prilagajanju glede na specializiran nabor podatkov lahko modeli opravljajo naloge, kot so dopolnjevanje kode in načrtovanje strojne opreme.  Najsodobnejši modeli imajo na milijarde parametrov, ki jih je mogoče natrenirati, in so učeni na več milijonih programskih repozitorijev. V tem delu se bomo osredotočili na modele ChatGPT, GitHub Copilot in Amazon Codewhisperer ter jih medsebojno primerjali.
V naslednjem poglavju bomo najprej pregledali obstoječo literaturo povezano z uporabo UI s poudarkom na hitrem inženiringu in razvoju programske opreme s pomočjo UI. 
\cite{zhang2003machine}
\section{Sorodna literatura}

\cite{app13095783}

Med že objavljenimi deli so bile raziskane priložnosti in izzivi uporabe ChatGPT v izobraževanju. Ugotovili so, da ChatGPT ponuja velike prednosti, kot so omogočanje personaliziranega učenja in zagotavljanje takojšnjih povratnih informacij, vendar prinaša tudi izzive, vključno z možnostjo prekomerne odvisnosti od UI in potrebo po splošni digitalni pismenosti med študenti.
Uporaba modelov, ki temeljijo na UI, kot je ChatGPT, v računalniškem programiranju je obširno raziskana. Ti modeli so uspešni pri nalogah, kot so dopolnitve kode, povzemanje, dokončanje, popravljanje, razvrščanje in generiranje kode. Na primer, AlphaCode, napreden jezikovni model, je pokazal znatne zmožnosti pri različnih programerskih nalogah, kar poudarja potencial ChatGPT pri pomoči pri izobraževanju programiranja. \cite{rudolph2023war}
\newline
V delu \textcite{MORADIDAKHEL2023111734} je raziskano programiranje s programom GitHub Copilot, kjer se primerja kvaliteta in pravilnost rešitev, ki jih le ta ponuja.
Sorodno delo je tudi \textcite{yetistiren2023evaluating}, kjer je narejena primerjava GitHub Copilot, Amazon CodeWhisperer in ChatGPT. V delu se omenjena orodja primerja, ter raziskuje pravilnost, kvaliteta, varnost in zanesljivost generirane kode za vsako orodje.
Veliko del je bilo narejenih z namenom preučevanja kvalitete in pravilnosti kode, ki jo orodja UI generirajo. Pri teh študijah se je potrebno zavedati, da se orodja UI stalno in zelo hitro razvijajo in lahko rezultati meril uspešnosti, ki so bili pridobljeni v preteklosti, ne držijo več za novejše verzije orodij.
Pri iskanju sorodne literature je bilo najdenih veliko del, ki preučujejo generirano kodo orodij UI. Ni pa bila še narejena študija, ki bi primerjala hitrost in pravilnost programiranja začetnikov ob souporabi ChatGPT napram njihovi kakovosti programiranja brez orodij. V naslednjih poglavjih bomo preučili razvoj orodij UI ter najpopularnejše predstavnike tudi primerjali.

Namen programskega inženirstva je poenostaviti proces razvoja programske opreme in ga narediti bolj učinkovitega. V naslednjem poglavju se bomo osredotočili na fazo implementacije in generiranja kode življenjskega cikla razvoja programske opreme. Pogledali bomo inženiring pozivov oziroma hiter inženiring ter analizirali njegov vpliv na kakovost programskega inženirstva.

Za analizo vpliva bomo temeljili na delu \cite{yetistiren2023evaluating}, kjer se primerja najpopularnejše asistente na področju programiranja z umetno inteligenco: ChatGPT, GitHub Copilot ter Amazon CodeWhisperer. Generirano kodo orodij se v omenjenem delu primerja z uporabo referenčne zbirke podatkov t.i. "HumanEval" ter oceni na podlagi predlaganih meril kakovosti kode. V nadaljevanju bomo uporabili podobna merila kakovosti kode, namesto same kode, ki jo generira chatGPT pa bomo analizirali kvaliteto kode pri začetnih programerjih kadar si pri pisanju kode pomagajo s chatGPT v primerjavi s kodo, napisano brez kakršnekoli asistence. Tako izpostavimo vpliv inženiringa pozivov pri generiranju kode pri razvoju programske opreme.


\cite{7577432}


\chapter{Hiter inženiring}

Hiter inženiring je razmeroma nova tehnika na področju obdelave naravnega jezika, ki zajema načrtovanje in optimizacijo pozivov, ki se uporabljajo za vnos informacij v modele UI, da bi izboljšali njihovo učinkovitost pri določenih nalogah. \cite{wang2024prompt} Gre za prakso učinkovitega sodelovanja s sistemi UI za optimizacijo njihovih koristi. Z nedavnim napredkom velikih jezikovnih modelov se je hitro inženirstvo izkazalo za pomembno in učinkovito na različnih področjih, vse od programiranja in učenja, do področja financ in zdravstva.
\cite{info:doi/10.2196/50638}

\section{Veliki jezikovni modeli}
Veliki jezikovni modeli so vrsta umetne inteligence, zasnovana za razumevanje in ustvarjanje besedila, ki je blizu ljudem. Treniranje temelji na vnaprej naučenih vzorcih z obsežnim usposabljanjem na veliki količini podatkov, kot so spletne strani, knjige in članki. S pomočjo podatkov lahko model prepozna statistične vzorce in strukture človeškega jezika, kar omogoča, da se nauči slovnice, sintakse in semantične odnose v jeziku. Po končanem učenju se model navadno usmeri in prilagodi za določeno domeno ali nalogo, kot je na primer prevajanje, generiranje besedila, prepoznavanje slik, ipd.
\cite{10.1145/3520312.3534862}
V zadnjih letih je prišlo do velikega napredka na področju procesiranja naravnih jezikov (PNJ) predvsem z razvojem arhitekture transformatorjev in mehanizma pozornosti.
Transformator je arhitektura globokega učenja, ki jo je razvil Google in temelji na mehanizmu večglave pozornosti, predstavljenem v članku "Attention Is All You Need" iz leta 2017.  \cite{datacamp_attention_2024} 
Besedilo se pretvori v numerične reprezentacije, imenovane žetoni, vsak žeton pa se s pomočjo tabele za vstavljanje besed pretvori v vektor. Na vsaki plasti se nato vsak žeton kontekstualizira z drugimi žetoni prek vzporednega mehanizma večglave pozornosti, ki omogoča, da se signal za ključne žetone okrepi, manj pomembni žetoni pa se zmanjšajo. To modelu omogoča boljše razumevanje odnosov med besedami v stavku, ne glede na njihov položaj.
Rekurentne enote so komponente v nekaterih nevronskih mrežah, ki si zapomnijo prejšnje informacije, da lahko pri obdelavi novega podatka upoštevajo kontekst. Vendar pa zaradi tega potrebujejo več časa za treniranje, ker morajo vsakič obdelati informacije korak za korakom.
Prednost transformatorjev je, da nimajo rekurentnih enot, kar jim omogoča, da obdelujejo podatke hitreje in bolj učinkovito. Poleg transformatorjev je pomembne spremembe prinesel tudi mehanizem pozornosti.
\cite{NIPS2017_3f5ee243}

Mehanizem pozornosti za razliko od tradicionalnih metod, ki besede obravnavajo ločeno, vsaki besedi dodeli uteži glede na njeno pomembnost za trenutno nalogo. To modelu omogoča, da zajame odvisnosti dolgega dosega, hkrati analizira lokalne in globalne kontekste ter razrešuje dvoumnosti s pozornostjo do informativnih delov stavka. Razvoj tega mehanizma je doprinesel k razvoju velikih jezikovnih modelov.
\cite{datacamp_attention_2024}
\cite{KASNECI2023102274}

Druga pomembna novost je uporaba učenja vnaprej, pri katerem se jezikovni model najprej uči na velikem naboru podatkov, nato pa se prilagodi določeni nalogi. To se je izkazalo za učinkovito tehniko za izboljšanje učinkovitosti pri številnih jezikovnih nalogah.
Nedavni napredek vključuje tudi razvoj najpopularnejšega modela ChatGPT, ki je bil učen na veliko večji zbirki podatkov, tj. besedilih iz zelo velikega spletnega korpusa, in je pokazal vrhunsko uspešnost pri številnih nalogah v naravnem jeziku, od prevajanja do odgovarjanja na vprašanja, pisanja koherentnih esejev in računalniških programov.
\cite{KASNECI2023102274}

Čeprav je razvoj velikih jezikovnih modelov zelo napredoval, je še vedno veliko omejitev, ki jih je treba odpraviti. Ena glavnih omejitev je pomanjkanje razlage odločitev, saj je težko razumeti razloge za napovedi modela. Obstajajo tudi etični pomisleki, kot so pomisleki glede pristranskosti in vpliva teh modelov, npr. na zaposlovanje, tveganje zlorabe in neustrezne ali neetične uporabe, izguba integritete in številni drugi. Na splošno bodo veliki jezikovni modeli še naprej premikali meje mogočega pri obdelavi naravnega jezika. Vendar je treba opraviti še veliko dela v smeri obravnavanja njihovih omejitev in s tem povezanih etičnih vidikov.

Jezikovni modeli dodeljujejo verjetnosti zaporedjem žetonov in se pogosto uporabljajo za besedila v naravnem jeziku. V zadnjem času so jezikovni modeli pokazali učinkovitost tudi pri modeliranju izvorne kode, napisane v programskih jezikih. Ti modeli se izkažejo pri nalogah, kot sta dopolnjevanje kode ter sintetizacija kode iz opisov v naravnem jeziku. 
Trenutni najsodobnejši veliki jezikovni modeli za kodo so pokazali pomemben napredek pri pomoči pri programiranju z asistenco umetne inteligence. Najbolj izstopa eden največjih modelov, Codex, ki se kot orodje GitHub Copilot uporablja kot pomočnik za razvijalce znotraj IDE, ki samodejno ustvarja kodo na podlagi uporabnikovega vnosa. V naslednjih poglavjih bomo analizirali model Codex ter orodje GitHub Copilot, ChatGPT ter orodje Amazon CodeWhisperer.

\cite{KASNECI2023102274}
\cite{vaswani2023attention}

\section{ChatGPT}

ChatGPT je jezikovni model, ki ga je 30. novembra 2022 objavila organizacija OpenAI in je na voljo od 1. februarja 2023. \cite{openai_codex}
ChatGPT uporablja napredne algoritme strojnega učenja za ustvarjanje človeku podobnih odgovorov. Treniran je na obsežni količini tekstovnih besedil iz interneta. Glavni namen uporabe jezikovnega modela je imitiranje človeškega pogovora, kljub temu pa je zelo vsestranski in lahko opravlja številne naloge, kot so kodiranje in odpravljanje hroščev v programski opremi, iskanje odgovorov na izpitna
vprašanja, sestavljanje pesniških del in glasbenih besedil ter mnogo drugih nalog. ChatGPT je bil posebej prilagojen na podlagi modela iz serije GPT-3.59, ki je v začetku leta 2022 zaključil svoj proces usposabljanja z uporabo nadzorovanega učenja, kot tudi učenja z okrepitvijo. Poleg tega organizacija OpenAI še naprej zbira povratne informacije iz pogovorov z ChatGPT, da bi izboljšali in izpopolnili njegovo delovanje.
Omembe vredno je tudi, da je ChatGPT postal najhitreje rastoča aplikacija v zgodovini
glede na študijo švicarske banke Union 'Bank of Switzerland'. Januarja 2023 je ChatGPT dnevno obiskalo 13 milijonov unikatnih obiskovalcev, kar je več kot dvakrat toliko kot v istem letu decembra, kot je pokazala študija. Poročilo navaja tudi, da je v prvih dveh mesecih ChatGPT dosegel mesečno bazo uporabnikov 100 milijonov. \cite{yetistiren2023evaluating}

Napredek na področju UI, zlasti velikih jezikovnih modelov, je močno vplival na številna področja, vključno z izobraževanjem in razvojem programske opreme. ChatGPT je postal pomembno orodje tako v izobraževalnih okoljih kot tudi v praktičnih programerskih aplikacijah. To poglavje obravnava vlogo ChatGPT v hitrem inženiringu in razvoju programske opreme, ocenjuje njegove prednosti, izzive in prihodnje vplive.
Hiter inženiring vključuje oblikovanje vhodov za modele UI, da bi dosegli želene odzive. Kot model UI, usposobljen na obsežnih podatkovnih zbirkah, je ChatGPT pokazal izjemno sposobnost interpretacije in generiranja besedil, podobnih človeškim, na podlagi različnih pozivov. Ta sposobnost je neprecenljiva v izobraževalnih okoljih, kjer lahko pomaga pri ustvarjanju interaktivnih učnih okolij, zagotavljanju takojšnjih povratnih informacij in generiranju učne vsebine. Na primer, ChatGPT lahko pomaga študentom pri razlagi zapletenih konceptov na enostavnejši način, s čimer izboljša njihovo učno izkušnjo.
Na področju razvoja programske opreme ChatGPT nudi veliko podporo pri generiranju kode, odpravljanju napak in optimizaciji. Orodja, kot so GitHub Copilot in Amazon CodeWhisperer, ki jih poganjajo modeli UI, podobni ChatGPT, pomagajo razvijalcem pri generiranju delov kode na podlagi opisov v naravnem jeziku. Ta funkcionalnost ne samo pospešuje proces kodiranja, temveč tudi pomaga začetnikom pri razumevanju strukture in sintakse kode.
Integracija ChatGPT v izobraževalne sisteme predstavlja številne priložnosti in izzive. Pozitivno je, da ChatGPT lahko demokratizira dostop do izobraževanja, saj zagotavlja prilagojeno mentorstvo in pomoč študentom po vsem svetu. Omogoča interaktivne učne izkušnje in pomaga pri razumevanju zapletenih predmetov, kot je programiranje. Raziskave so pokazale, da študenti, ki uporabljajo ChatGPT za programerske naloge in reševanje problemov, dosegajo boljše rezultate in razumevanje.

Vendar pa uporaba ChatGPT prinaša tudi tveganja, vključno z morebitnim prekomernim zanašanjem na UI za reševanje problemov, kar lahko ovira razvoj kritičnega mišljenja. Poleg tega je potrebno nenehno preverjanje natančnosti in zanesljivosti vsebine, ki jo generira UI, da bi preprečili širjenje napačnih informacij.
\cite{app13095783}
\section{GitHub Copilot}

GitHub Copilot je napredno orodje za generiranje kode v realnem času, ki se uporablja znotraj različnih integriranih razvojnih okolij (angl. IDE), kot so Visual Studio Code, Neovim, okolja JetBrains in GitHub Codespaces. GitHub Copilot neprekinjeno zbira uporabniške vnose kode in pošilja izseke v osnovni model OpenAI Codex podjetja OpenAI. Copilot generira kodo in predstavi rezultate modela OpenAI Codex tako, da predlagano generirano kodo prilagodi trenutni napisani kodi programerja znotraj IDE. Copilot je zmožen generirati kodo na podlagi opisa programskega problema v naravnem jeziku, kar omogoča reševanje nalog z generiranjem ustrezne kode. Poleg tega zna Codex kodo razložiti, jo prevesti med programskimi jeziki, napisati dokumentacijo, ter poiskati in popraviti napake v kodi.
Pomembna razširitev Copilota je GitHub Copilot Chat (GCC) , ki je bil objavljen 29. 12. 2023 in temelji na modelu GPT-4 podjetja OpenAI namesto modela Codex. Codex je starejši model, ki izhaja iz GPT-3 modela, ki je bil narejen v sodelovanju s podjetjem Github posebej za generiranje kode. GPT-4 ni razvit posebej za kodiranje, vendar je močno izboljšan model v primerjavi z modelom GPT-3.
\cite{Sundqvist1866649}
GCC znotraj IDE urejevalnika doda pogovorno okno, kjer lahko razvijalec v poljubnem naravnem jeziku postavlja vprašanja, zahteva razlago ali popravke kode, ter celo generiranje dokumentacije ali testov, brez da bi zapustil IDE. Poleg poziva, ki ga podamo v pogovorno okno GCC upošteva tudi kontekst trenutno odprte datoteke urejevalnika, dodatno pa lahko v kontekst pogovora v urejevalniku odprtega projekta referenciramo poljubne datoteke, ki se nanašajo na naše vprašanje, da dobimo bolj natančen odgovor.
\cite{github_copilot_chat}
\newline
Copilot temelji na družini modelov OpenAI Codex. Modeli Codex kot osnovo vzamejo model GPT-3, ki ga nato prilagodijo na podlagi kode iz GitHuba. Njegov način tokenizacije je skoraj identičen z GPT-3, uporablja kodiranje parov bajtov za pretvorbo izvornega besedila v zaporedje žetonov, vendar je bil besedni zaklad GPT-3 razširjen z dodajanem namenskih žetonov za bele prostore (tj. žeton za dva presledka, žeton za tri presledke, vse do 25 presledkov). To tokenizerju omogoča učinkoviteje kodiranje izvorne kode. (ki ima veliko belih presledkov) Pomembna lastnost, ki sta jo Codex in Copilot podedovala od GPT-3, je, da ob pozivu ustvarita najverjetnejši odgovor za poziv na podlagi tega, kar je bilo pridobljeno iz izkušenj, pridobljenih pri usposabljanju. V kontekstu generiranja kode to pomeni, da model ne bo nujno ustvaril najboljše kode (po kateri koli izbrani metriki - zmogljivost, varnost itd.), temveč tisto, ki se najbolje ujema s predhodno in naučeno kodo. Posledično na kakovost generirane kode močno vlivajo semantično nepomembne lastnosti poziva. \cite{9833571}
\cite{10.1145/3520312.3534862}



\section{Amazon CodeWhisperer}
Amazon CodeWhisperer izboljša produktivnost razvijalcev z generiranjem priporočil za kodo na podlagi komentarjev razvijalcev v angleščini in predhodne kode v IDE. AWS je napovedal Amazon CodeWhisperer 23. junija 2022. Predlogi kode, ki jih ponuja temeljijo na modelih strojnega učenja, ki so bili naučeni na različnih Amazonovih podatkovnih virih ter z odprtokodnimi repozitoriji. Ko
razvijalci napišejo komentar v urejevalniku kode v svojem IDE, CodeWhisperer samodejno pregleda komentar in določi najprimernejše storitve v oblaku in javne knjižnice. Nato bo neposredno v urejevalniku kode vrnil odlomek kode. CodeWhisperer razvijalcem dodatno poenostavi uporabo storitev AWS, saj ponuja predloge za AWS API kodo v storitvah, kot je Amazon Elastic Compute Cloud (EC2), AWS Lambda in Amazon Simple Storage Service (S3). CodeWhisperer podpira več IDE, vključno z JetBrains, Visual Studio Code, AWS Cloud9 ali konzolo AWS Lambda kot del nabora orodij AWS IDE. Trenutno podpira Javo, JavaScript, Python, C\# in Typescript. Kot dodatno funkcijo ima CodeWhisperer sledilnik referenc, ki zazna priporočila za kodo podobne določenim podatkom za usposabljanje CodeWhispererja in ta priporočila posreduje razvijalcem. CodeWhisperer zna kodo tudi pregledati ter opredeliti varnostna tveganja.

\section{Ugotovitve in primerjava}

\begin{table}[H]
\centering
\caption{Primerjava orodij ChatGPT, GitHub Copilot in Amazon CodeWhisperer}
\begin{tabular}{|>{\raggedright\arraybackslash}p{3cm}|>{\raggedright\arraybackslash}p{4cm}|>{\raggedright\arraybackslash}p{4cm}|>{\raggedright\arraybackslash}p{4cm}|}
\hline
\textbf{Lastnost}               & \textbf{ChatGPT}           & \textbf{GitHub Copilot}           & \textbf{Amazon CodeWhisperer}          
\\ \hline
Razvijalec              & OpenAI      & OpenAI-Microsoft                  & AWS            \\ \hline
Datum izdaje               & 30. 11. 2022        & 29. 10. 2021           & 23. 06. 2022       \\ \hline
Podpora integriranega razvojnega okolja           & Ni podpore          & Večina razvojnih okolij        & Pomembnejša razvojna okolja          \\ \hline
Razlaga predlagane kode       & Da                       & Da (omogoča GitHub Copilot Chat)                           & Ne                       \\ \hline
Navajanje sklicev na predloge                  & Ne                      & Ne                    & Da                      \\ \hline
Več predlogov za poziv                & Ne (možno z eksplicitnimi navodili za več predlogov)         & Da (do 10)                  & Da (do 5)               \\ \hline
Vir podatkov za usposabljanje             & GitHub repozitoriji, zbirka podatkov OpenAI Codex, online repozitoriji GitLab, Bitbucket in SourceForge                   & "Velike količine javno objavljene kode"                  & "...treniran na vseh jezikih iz javnih repozitorijev"                    \\ \hline
S katerimi jeziki deluje najboljše                        & Ni zapisano                  & C, C++, C\#, Go, Java, JavaScript, PHP, Python, Ruby, Scala in TypeScript              & C\#, Java, JavaScript, Python in TypeScript  \\ \hline
Cena               & osnovni model zastonj, plus 20\$ / mesec          & Zastonj za študente in učitelje, 10\$ / mesec za posameznike, 19\$ / mesec za profesionalne uporabnike                    &   Zastonj za posameznike, 19\$ / mesec za profesionalne uporabnike            \\ \hline
\end{tabular}
\label{tab:comparison}
\end{table}

Iz javno dostopnih podatkov ter testiranja zgoraj najpopularnejših predhodno omenjenih orodij smo naredili zgornjo tabelo, ki primerja modele ChatGPT, GitHub Copilot ter Amazon CodeWhisperer. Integracijo v razvojno okolje podpirata tako GitHub Copilot kot tudi Amazon CodeWhisperer, kjer GitHub Copilot ponuja bolj obširno izbiro razvojnih orodij. Razlago predlagane kode podpirata ChatGPT in GitHub Copilot Chat, sklice na predlagane rešitve pa ponuja le Amazon CodeWhisperer. Največje število predlogov za podan poziv ponuja Github Copilot, pri uporabi ChatGPT pa lahko to dosežemo z eksplicitnimi navodili. ChatGPT nima podanih informacij o najbolje podprtih programskih jezikih, GitHub Copilot in Amazon CodeWhisperer navajata podprte programske jezike, pri čemer lahko vidimo, da GitHub Copilot podpira več programskih jezikov kot Amazon CodeWhisperer. Pomembna razlika v modelih je tudi v ceni, saj nižja cena pripomore k različni dostopnosti širši množici uporabnikom, kar je eden izmed razlogov za popularnost ChatGPT-ja tudi v programiranju, kljub temu, da to ni njegov prvotni namen. Kljub temu, da je osnovni model zastonj, pa lahko plačamo 20\$ na mesec za napredni model, ki uporablja GPT-4. Amazon CodeWhisperer ima tudi ugoden cenovni načrt z možnostjo brezplačne uporabe za posameznike in ceno 19\$ mesec za profesionalne uporabnike. GitHub Copilot ponuja brezplačno uporabo le za učitelje in študente, ostali uporabniki plačujejo 10\$ / mesec ali pa 19\$ / mesec, če gre za profesionalne uporabnike.
\cite{github_copilot_chat}
\cite{openai_chatgpt}
\cite{saasworthy_codewhisperer}

Za izvedbo eksperimenta vpliva hitrega inženiringa pri razvoju programske opreme smo si glede na zgoraj opisane podatke izbrali ChatGPT, predvsem zaradi splošne poznanosti in popularnosti orodja, saj je velika večina začetnih programerjev že imela interakcijo z orodjem pred samim eksperimentom.

\section{Programiranje s ChatGPT}
Izmed predhodno izbranimi in opisanimi asistenti umetne inteligence je zaradi preproste in široke možnosti uporabe najbolj dostopen in popularen ChatGPT podjetna OpenAI. Velik vpliv na popularnost ima dejstvo, da orodje lahko uporabljamo preko preprostega spletnega vmesnika, ki je v osnovni verziji neplačljiv.

ChatGPT se lahko kot izobraževalni vir programiranja uporablja kot interaktivni inštruktor, ki odgovarja na vprašanja o uporabi programskega jezika, sintaksi in semantiki kode, najboljših praksah, razpoložljivih knjižnicah ali paketih, alternativnih pristopih, integriranih razvojnih okoljih (IDE) in programskih okoljih. ChatGPT lahko ustvari tudi preprost/jasen primer kode s komentarji za vsako vrstico in povzetkom v naravnem jeziku o tem, kaj koda počne (s poudarjeno osnovno funkcijo ključnih spremenljivk, metod ali vključenih paketov). V nasprotju z uporabo iskalnika Google ali spletnih mest, kot so stackoverflow, geeksforgeeks, za vprašanja o programiranju, lahko ChatGPT učencu neposredno ponudi preprosto, razumljivo in ustrezno rešitev za določeno vprašanje o programiranju. Učitelji lahko ChatGPT zadolžijo tudi za ustvarjanje učnih gradiv za kodiranje ter domačih nalog in izpitnih vprašanj, povezanih z določenimi temami programiranja.
Poleg pisanja nove kode lahko ChatGPT opravlja številne dodatne funkcije za izboljšanje obstoječe kode. Izmenjava kode na platformah, kot je GitHub, ponuja izjemne možnosti za pospešitev raziskav in razvoja na področju računalništva. Vendar je lahko težko interpretirati kodo, ki so jo napisali drugi (npr. študenti ali celo naša lastna koda iz preteklih let), saj pogosto obstaja veliko ustreznih načinov za programiranje posamezne naloge. To še posebej velja, kadar je koda slabo dokumentirana z omejenimi, netočnimi ali dvoumnimi komentarji. Uporabnik lahko ChatGPT vpraša, kaj določena vrstica ali del kode počne, in ChatGPT jo bo poskušal razčleniti na posamezne dele, pojasniti spremenljivke, ukaze in korake ter vključiti splošen povzetek tega, kar po njegovem mnenju ta koda počne. Pri sorodni nalogi lahko ChatGPT prosimo, da doda ali popravi komentarje v kodi, s čimer se avtomatizira dokumentiranje kode. ChatGPT lahko olajša tudi odpravljanje napak v kodi, čeprav bo, tako kot pri ročnem kodiranju, napake, ki vplivajo na pričakovano delovanje ali zmogljivost kode, verjetno veliko težje prepoznati kot tiste, ki preprosto preprečujejo uspešno delovanje kode. Poleg tega lahko uporabniki od programa zahtevajo, da poskuša poenostaviti obstoječo kodo, da bi jo naredil bolj kompaktno, razumljivo ali računsko učinkovito, ali pa da kodo prevede iz enega programskega jezika v drugega.

V skladu z opozorili za splošno uporabo lahko ChatGPT nekoliko nepredvidljivo prikaže napačno kodo kot pravilno. Poleg tega morda ne pozna ali ne more predvideti robnih primerov, ki bi lahko v posebnih okoliščinah porušili funkcionalnost kode, in morda privzeto ne predstavi najboljše ali najučinkovitejše rešitve kodiranja. Na koncu lahko koda, ki jo napiše ChatGPT, ponudi koristno izhodišče, vendar morajo programerji in uporabniki preveriti in potrditi kodo in komentarje, ki jih ustvari, da se prepriča, da v celoti izpolnjujejo predvideni namen. Neposredno zanašanje na kodo, ki jo ustvari ChatGPT, bi bilo slabo premišljeno pri aplikacijah z visokim tveganjem, kjer so varnost, odgovornost, zasebnost in zaupanje najpomembnejši. Tako se vsaj za zdaj zdi, da chatGPT ne more nadomestiti izkušenih programerjev, lahko pa pri olajša in pospeši nekatere procese pri razvoju programske opreme.
\cite{Meyer2023}

Ekipa OpenAI je napisala 164 težav s testi, da bi ocenila, ali
zgodnji modeli Codexa ustvarjajo pravilno kodo Python iz angleščine. Skoraj 29 \% problemov je bilo rešenih s Codexovim prvim predlogom. 
Natančno prilagojen model, usposobljen samo na pravilni kodi Pythona,
je v prvem poskusu rešil 38 \% problemov. Če je ta model
dovoljeno 100 poskusov na problem, potem je bil vsaj eden od njih
pravilen pri 78 \% problemov.

\cite{DBLP:journals/corr/abs-2107-03374}

V nadaljevanju bomo s pomočjo nadzorovanega eksperimenta preučili, do kakšne mere vpliva uporaba chatGPT-ja na hitrost in kakovost programiranja pri programerjih začetnikih.


\chapter{Metodologija}
Namen diplomskega dela je bil raziskati, kako hiter inženiring vpliva na proces razvoja programske opreme, s preučevanjem hitrosti in kvalitete kode pri uporabi orodij umetne inteligence. Najprimernejša metoda za pridobivanje teh podatkov je bil nadzorovan eksperiment na skupini mlajših otrok, ker omogoča natančen nadzor nad spremenljivkami ter izolirano vrednotenje reševanja programskih problemov. Ta metodologija minimizira vplive zunanjih spremenljivk, kar omogoča osredotočeno preučevanje zmožnosti in omejitev hitrega inženiringa pri razvoju programske opreme.

\section{Načrt}

\begin{figure}[H]
    \centering
    \includegraphics[width=1\linewidth]{spletna_stran.png}
    \caption{Spletna stran eksperimenta}
    \label{fig:enter-label}
\end{figure}


Za izvedbo eksperimenta smo izbrali več skupin mlajših otrok, ki se učijo programiranja v jeziku Python. Za namen eksperimenta je bila razvita preprosta spletna stran, ki je imela navodila za različne naloge, prostor za pisanje Python kode z označevanjem sintakse, prostor za izpis izhoda programa ter prostor za pogovor z uporabo ChatGPT-3.5. Namen strani je bil anonimizirano shranjevanje kode med eksperimentom, za namen kasnejšega preučevanja. Stran je kodo shranila ob vsakem kliku na gumb "Run" za izvajanje napisane kode ter na določen časovni interval. Shranjevalo se je napisano kodo, čas zapisa, število poskusov za trenutno nalogo ter če se je trenutna koda izvedla z napako. Eksperiment je trajal dve uri z vmesnih petnajst minutnim odmorom in je bil sestavljen iz osmih nalog. Vrstni red reševanja nalog je bil naključen brez možnosti vračanja na predhodne naloge, da bi zmanjšali prepisovanje. Za polovico nalog je bila uporaba chatGPT onemogočena, za lažjo analizo podatkov. Naloge so bile razdeljene v tri skupine težavnosti: lahke, srednje težke in težke. Težavnost naloge je bila dodeljena po pogovoru z ostalimi učitelji Pythona, glede na potrebno znanje in časovno zahtevnost naloge.

\begin{figure}[H]
    \centering
    \includegraphics[width=1\linewidth]{naloga_primer.png}
    \caption{Primer naloge}
    \label{fig:enter-label}
\end{figure}
\section{Izvedba}
Eksperiment je bil izpeljan v štirih skupinah, ena izmed teh je bila v fizični obliki, ostale skupine pa so bile na daljavo. Skupaj je bilo veljavnih 32 podatkov. Eksperiment je potekal tako, da je bilo najprej predstavljena umetna inteligenca in hiter inženiring, ter najboljše prakse za uporabo ChatGPT za pomoč pri programiranju. Nato je bila rešena krajša anketa o predznanju programiranja ter o poznavanju ChatGPT-ja. Po tem so imeli dobro uro za samo reševanje nalog. Med samim eksperimentom se otrokom ni pomagalo, razen z interpretacijo navodil.
\section{Ocena}
Znotraj enake težavnosti nalog so bile naloge, pri katerih je bila omogočena uporaba pogovornega okna s ChatGPT rešene mnogo hitreje in z manj napakami. 

\printbibliography




\end{document}
