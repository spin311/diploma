\documentclass[12pt,a4paper]{book}

%Uporabljeni paketi
\usepackage[utf8]{inputenc}
\usepackage{cmap}
\usepackage{type1ec}
\usepackage[T1]{fontenc}
\usepackage{fancyhdr}
\usepackage{graphicx,epsfig}
\usepackage[slovene]{babel}
\usepackage{cite}

\usepackage[pdftex,colorlinks,citecolor=black,filecolor=black,linkcolor=black,urlcolor=black,pagebackref]{hyperref}
\usepackage{tikz}

%Velikost strani - dvostransko
\oddsidemargin 1.4cm
\evensidemargin 0.35cm
\textwidth 14cm
\topmargin 0.26cm
\headheight 0.6cm
\headsep 1.5cm
\textheight 20cm

%Nastavitev glave in repa strani
\pagestyle{fancy}
\fancyhead{}
\renewcommand{\chaptermark}[1]{\markboth{\textsf{Poglavje \thechapter:\ #1}}{}}
\renewcommand{\sectionmark}[1]{\markright{\textsf{\thesection\  #1}}{}}
\fancyhead[RE]{\leftmark}
\fancyhead[LO]{\rightmark}
\fancyhead[LE,RO]{\thepage}
\fancyfoot{}
\renewcommand{\headrulewidth}{0.0pt}
\renewcommand{\footrulewidth}{0.0pt}

\newcommand{\gnuplot}{\textbf{gnuplot}}
\newcommand{\pgfname}{\textsc{pgf}}
\newcommand{\tikzname}{Ti\emph{k}Z}

\begin{document}

\thispagestyle{empty} 
\begin{center}
{\large 
UNIVERZA V LJUBLJANI\\
FAKULTETA ZA RAČUNALNIŠTVO IN INFORMATIKO\\
}

\vspace{3cm}
{\LARGE Svit Spindler}\\

\vspace{2cm}
\textsc{\textbf{\LARGE 
Hiter inženiring pri razvoju programske opreme\\ 
}}

\vspace{2cm}
{ DIPLOMSKO DELO}\\
{ NA UNIVERZITETNEM ŠTUDIJU\\
}

\vspace{2cm} 
{\Large Mentor: izr. prof. dr. Dejan Lavbič}

\vfill
{\Large Ljubljana, 2024}
\end{center}

\newpage

%********************************************

% stran 3 med uvodnimi listi
\thispagestyle{empty}

Namesto te strani {\bf vstavite} original izdane teme diplomskega dela s podpisom mentorja in dekana ter \v zigom fakultete, ki ga diplomant
dvigne v študent\-skem referatu,  preden odda izdelek v vezavo!

\newpage

\newpage

\ \thispagestyle{empty}

\newpage

%********************************************

% stran 2 med uvodnimi listi
\thispagestyle{empty}

\vspace*{5cm}
{\small \noindent
To diplomsko delo je ponujeno pod licenco \textit{Creative Commons Priznanje avtorstva-Deljenje pod enakimi pogoji 2.5 Slovenija}
ali (po želji) novejšo različico.
To pomeni, da se tako besedilo, slike, grafi in druge sestavine dela kot tudi rezultati diplomskega dela lahko prosto distribuirajo,
reproducirajo, uporabljajo, dajejo v najem, priobčujejo javnosti in predelujejo, pod pogojem, da se jasno in vidno navede avtorja in naslov tega
dela in da se v primeru spremembe, preoblikovanja ali uporabe tega dela v svojem delu, lahko distribuira predelava le pod
licenco, ki je enaka tej.
Podrobnosti licence so dostopne na spletni strani \url{http://creativecommons.si/} ali na Inštitutu za
intelektualno lastnino, Streliška 1, 1000 Ljubljana.

\begin{center}% 0.66 / 0.89 = 0.741573033707865

\end{center}
}

\vspace*{1.5cm}
{\small \noindent
Izvorna koda diplomskega dela, njenih rezultatov in v ta namen razvite programske opreme je ponujena pod GNU General Public License,
različica 3 ali (po želji) novejšo različico. To pomeni, da se lahko prosto uporablja, distribuira in/ali predeluje pod njenimi pogoji.
Podrobnosti licence so dostopne na spletni strani \url{http://www.gnu.org/licenses/}.
}

\begin{center}
\ \\ \vfill
{\em
Besedilo je oblikovano z urejevalnikom besedil \LaTeX. \\ Slike so izdelane s pomočjo jezika \pgfname/\tikzname. \\ Grafi so narisani
s pomočjo programa \gnuplot.}
\end{center}

\newpage

\ \thispagestyle{empty}

\newpage

%********************************************

% stran 3 med uvodnimi listi
\thispagestyle{empty}

Namesto te strani {\bf vstavite} original izdane teme diplomskega dela s podpisom mentorja in dekana ter \v zigom fakultete, ki ga diplomant
dvigne v študent\-skem referatu,  preden odda izdelek v vezavo!

\newpage

%********************************************

% stran 4 med uvodnimi listi je prazna
\ \thispagestyle{empty}

\newpage

%********************************************

% stran 5 med uvodnimi listi

\thispagestyle{empty}

\vspace{1cm}
\begin{center}
{\Large \textbf{IZJAVA O AVTORSTVU}}
\end{center}

\begin{center}
{\Large diplomskega dela}
\end{center}

\vspace{1cm}
Spodaj podpisani/-a \hspace{0.5cm} Ime Priimek,

\vspace{0.5cm}
z vpisno številko \hspace{0.5cm} xxxxxxx,

\vspace{1cm}
sem avtor/-ica diplomskega dela z naslovom:

\vspace{0.5cm}
Naslov diplomskega dela

\vspace{1.5cm}
S svojim podpisom zagotavljam, da:
\begin{itemize}
	\item sem diplomsko delo izdelal/-a samostojno pod mentorstvom

	prof. [doc.] dr. Ime Priimek

	in somentorstvom

	prof. [doc.] dr. Ime Priimek

	\item	so elektronska oblika diplomskega dela, naslov (slov., angl.), povzetek (slov., angl.) ter ključne besede (slov., 			angl.) identični s tiskano obliko diplomskega dela
	\item soglašam z javno objavo elektronske oblike diplomskega dela v zbirki ''Dela FRI''.
\end{itemize}

\vspace{1cm}
V Ljubljani, dne xx.xx.20xx \hspace{1cm} Podpis avtorja/-ice:

\newpage

%********************************************

% stran 6 med uvodnimi listi je prazna pri dvostranskem tiskanju
\ \thispagestyle{empty}

\newpage
\ \thispagestyle{empty}

%********************************************

% stran 7 med uvodnimi listi

\chapter*{Zahvala}

\thispagestyle{empty}

Na tem mestu se diplomant zahvali vsem, ki so kakorkoli pripomogli k uspešni izvedbi diplomskega dela.
\\**Zahvala podjetju Digital School D.O.O, za pomoč pri analizi učenja otrok z uporabo umetne inteligence. **


\newpage

%********************************************

% stran 8 med uvodnimi listi je prazna pri dvostranskem tiskanju
\ \thispagestyle{empty}

\newpage

%********************************************

\renewcommand\thepage{}
\tableofcontents
\renewcommand\thepage{\arabic{page}}

\thispagestyle{empty}


%********************************************

\chapter*{Seznam uporabljenih kratic in simbolov}

\thispagestyle{empty}

Seznam uporabljenih kratic in simbolov, ki morajo biti enotni v celotnem delu, ne glede na označevanje v uporabljenih virih.

%\cleardoublepage

\clearpage{\pagestyle{empty}\cleardoublepage}

%********************************************
%zacno se glavni listi, ki so numerirani z arabskimi stevilkami

\setcounter{page}{1}
\pagenumbering{arabic}

\chapter*{Povzetek}

\addcontentsline{toc}{chapter}{Povzetek}

Povzetek naj posreduje bralcu kratko vsebino dela. Zajema naj namen dela, področje, na katerega se delo nanaša,
uporabljene metode, poglavitne rezultate dela, zaključke in priporočila.
Povzetek naj ne obsega več kot eno stran, obi\v cajno ima le 200 do 300 besed. Napiše se povsem na koncu,
ko je že jasna vsebina vseh ostalih poglavij.

Ta dokument vsebuje navodilo za izdelavo diplomskega dela v obliki in strukturi, ki je v teh navodilih predpisan za
pisanje diplomskih nalog. Struktura dokumenta je namenjena obojestranskemu tiskanju, kjer se novo poglavje vedno za\v cne na lihi strani.
V dejanski diplomi poglavja in podpogla\-vja  obi\v cajno niso tako kratka kot v teh navodilih.

Za oblikovanje tega dokumenta je bil uporabljen sistem \LaTeX.
Ve\v c o \LaTeX-u lahko izve\v s na spletni strani \texttt{http://www.ctan.org/}.
Kandidati, ki bodo svoje diplomsko delo oblikovali s pomo\v cjo
\LaTeX-a, lahko izvorno kodo tega dokumenta neposredno uporabijo kot vzorec za pisanje svoje diplomske naloge.

\vspace{1.3cm}
\noindent
{\large \bf Ključne besede:}

\vspace{0.5cm}
\noindent
diploma, mentor, zagovor, podaljšanje, pisanje, struktura


\chapter*{Abstract}

\addcontentsline{toc}{chapter}{Abstract}

Povzetek naj bo napisan v angleškem jeziku.

\vspace{1.3cm}
\noindent
{\large \bf Key words:}

\vspace{0.5cm}
\noindent
Ključne besede v angleškem jeziku.


%********************************************

\chapter{Uvod}

Umetna inteligenca je postala nepogrešljivo orodje pri programiranju. Orodja za generiranje kode s pomočjo umetne inteligence so vse bolj razširjena v inženirstvu programske opreme in ponujajo možnost generiranja kode na podlagi pozivov v naravnem jeziku ali delnih vnosov kode. Pomembni primeri teh orodij so GitHub Copilot, Amazon CodeWhisperer in ChatGPT podjetja OpenAI. Omenjena orodja uporabljajo tako začetni, kot tudi izkušenejši programerji in pripomorejo k hitrejšemu in učinkovitejšemu reševanju problemov, s katerimi se programerji soočajo.
Namen diplomskega dela je raziskati prednosti in slabosti hitrega inženiringa (angleško prompt engineering) v programskem inženirstvu ter preizkusiti uporabnost orodja ChatGPT podjetja OpenAI pri učenju programiranja.



\chapter{Pregled področja}
Veliki jezikovni modeli (angleško LLM) so vrsta umetne inteligence, zasnovana za razumevanje in ustvarjanje besedila, ki je blizu ljedem. Treniranje temelji na vnaprej naučenih vzorcih z obsežnim usposabljanjem na veliki količini podatkov, kot so spletne strani, knjige, članki. S pomočjo podatkov lahko model prepozna statistične vzorce in strukture človeškega jezika, kar omogoča, da se nauči slovnice, sintakse in semantične odnose v jeziku. Po končanem učenju se model navadno usmeri in prilagodi za določeno domeno ali nalogo, kot je na primer prevajanje, generiranje besedila, prepoznavanje slik, ipd.
\section{Asistenti na področju umetne inteligence}


\section{Hitro programiranje}

\section{Programiranje s ChatGPT}
Preučevanje prednosti in slabosti, ocena pravilnosti in razlage rešitve, primerjava z ostalimi  AI asistenti.

\section{Programiranje z GithubCopilot}
Preučevanje prednosti in slabosti, ocena pravilnosti in razlage rešitve, primerjava z ostalimi  AI asistenti.

\section{Ugotovitve in primerjava}

\chapter{Preučevanje hitrega programiranja na skupini otrok}
\section{Načrtovanje}

\section{Izpeljava}

\section{Ocena}

\chapter{Prijava diplomskega dela}





\chapter{Viri}
Za kakovostno diplomsko delo je pomembna uporaba vseh razpoložljivih doma\-čih in tujih strokovnih ter znanstvenih virov.

Yetiştiren, B., Özsoy, I., Ayerdem, M., \& Tüzün, E. (2023, April 21). Evaluating the Code Quality of AI-Assisted Code Generation Tools: An Empirical Study on GitHub Copilot, Amazon CodeWhisperer, and ChatGPT. Pridobljeno 28. 02. 2024 iz: https://arxiv.org/abs/2304.10778

Thorsten Brants, Ashok C. Popat, Peng Xu, Franz J. Och, and Jeffrey Dean. 2007. Large Language Models in Machine Translation. In Proceedings of the 2007 Joint Conference on Empirical Methods in Natural Language Processing and Computational Natural Language Learning (EMNLP-CoNLL), pages 858–867, Prague, Czech Republic. Association for Computational Linguistics.

\begin{thebibliography}{9}
\bibitem{yetistiren2023evaluating}
B.~Yetiştiren, I.~Özsoy, M.~Ayerdem, \& E.~Tüzün.
\newblock Evaluating the Code Quality of AI-Assisted Code Generation Tools: An Empirical Study on GitHub Copilot, Amazon CodeWhisperer, and ChatGPT.
\newblock Retrieved February 28, 2024, from \url{https://arxiv.org/abs/2304.10778} (April 21, 2023).

\bibitem{brants2007large}
Thorsten Brants, Ashok C. Popat, Peng Xu, Franz J. Och, \& Jeffrey Dean.
\newblock Large Language Models in Machine Translation.
\newblock In \textit{Proceedings of the 2007 Joint Conference on Empirical Methods in Natural Language Processing and Computational Natural Language Learning (EMNLP-CoNLL)}, pages 858--867, Prague, Czech Republic, 2007. Association for Computational Linguistics.

\bibitem{yong2023authors}
Yong Zheng.
\newblock Authors Info \& Claims.
\newblock In \textit{SIGITE '23: Proceedings of the 24th Annual Conference on Information Technology Education}, October 2023, pages 66--72, 2023. DOI: \url{10.1145/3585059.3611431}.
\end{thebibliography}



\end{document}
