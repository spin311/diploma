\documentclass[12pt,a4paper]{book}

%Uporabljeni paketi
\usepackage[utf8]{inputenc}
\usepackage{cmap}
\usepackage{type1ec}
\usepackage[T1]{fontenc}
\usepackage{fancyhdr}
\usepackage{graphicx,epsfig}
\usepackage[slovene]{babel}
\usepackage{cite}

\usepackage[pdftex,colorlinks,citecolor=black,filecolor=black,linkcolor=black,urlcolor=black,pagebackref]{hyperref}
\usepackage{tikz}

%Velikost strani - dvostransko
\oddsidemargin 1.4cm
\evensidemargin 0.35cm
\textwidth 14cm
\topmargin 0.26cm
\headheight 0.6cm
\headsep 1.5cm
\textheight 20cm

%Nastavitev glave in repa strani
\pagestyle{fancy}
\fancyhead{}
\renewcommand{\chaptermark}[1]{\markboth{\textsf{Poglavje \thechapter:\ #1}}{}}
\renewcommand{\sectionmark}[1]{\markright{\textsf{\thesection\  #1}}{}}
\fancyhead[RE]{\leftmark}
\fancyhead[LO]{\rightmark}
\fancyhead[LE,RO]{\thepage}
\fancyfoot{}
\renewcommand{\headrulewidth}{0.0pt}
\renewcommand{\footrulewidth}{0.0pt}

\newcommand{\gnuplot}{\textbf{gnuplot}}
\newcommand{\pgfname}{\textsc{pgf}}
\newcommand{\tikzname}{Ti\emph{k}Z}

\begin{document}

\thispagestyle{empty} 
\begin{center}
{\large 
UNIVERZA V LJUBLJANI\\
FAKULTETA ZA RAČUNALNIŠTVO IN INFORMATIKO\\
}

\vspace{3cm}
{\LARGE Svit Spindler}\\

\vspace{2cm}
\textsc{\textbf{\LARGE 
Hiter inženiring pri razvoju programske opreme\\ 
}}

\vspace{2cm}
{ DIPLOMSKO DELO}\\
{ NA UNIVERZITETNEM ŠTUDIJU\\
}

\vspace{2cm} 
{\Large Mentor: izr. prof. dr. Dejan Lavbič}

\vfill
{\Large Ljubljana, 2024}
\end{center}

\newpage

%********************************************

% stran 3 med uvodnimi listi
\thispagestyle{empty}

Namesto te strani {\bf vstavite} original izdane teme diplomskega dela s podpisom mentorja in dekana ter \v zigom fakultete, ki ga diplomant
dvigne v študent\-skem referatu,  preden odda izdelek v vezavo!

\newpage

\newpage

\ \thispagestyle{empty}

\newpage

%********************************************

% stran 2 med uvodnimi listi
\thispagestyle{empty}

\vspace*{5cm}
{\small \noindent
To diplomsko delo je ponujeno pod licenco \textit{Creative Commons Priznanje avtorstva-Deljenje pod enakimi pogoji 2.5 Slovenija}
ali (po želji) novejšo različico.
To pomeni, da se tako besedilo, slike, grafi in druge sestavine dela kot tudi rezultati diplomskega dela lahko prosto distribuirajo,
reproducirajo, uporabljajo, dajejo v najem, priobčujejo javnosti in predelujejo, pod pogojem, da se jasno in vidno navede avtorja in naslov tega
dela in da se v primeru spremembe, preoblikovanja ali uporabe tega dela v svojem delu, lahko distribuira predelava le pod
licenco, ki je enaka tej.
Podrobnosti licence so dostopne na spletni strani \url{http://creativecommons.si/} ali na Inštitutu za
intelektualno lastnino, Streliška 1, 1000 Ljubljana.

\begin{center}% 0.66 / 0.89 = 0.741573033707865
  \CcImageCc{0.741573033707865}\hspace*{1ex}\CcGroupBySa{1}{1ex}
\end{center}
}

\vspace*{1.5cm}
{\small \noindent
Izvorna koda diplomskega dela, njenih rezultatov in v ta namen razvite programske opreme je ponujena pod GNU General Public License,
različica 3 ali (po želji) novejšo različico. To pomeni, da se lahko prosto uporablja, distribuira in/ali predeluje pod njenimi pogoji.
Podrobnosti licence so dostopne na spletni strani \url{http://www.gnu.org/licenses/}.
}

\begin{center} 
\ \\ \vfill
{\em
Besedilo je oblikovano z urejevalnikom besedil \LaTeX. \\ Slike so izdelane s pomočjo jezika \pgfname/\tikzname. \\ Grafi so narisani
s pomočjo programa \gnuplot.}
\end{center}

\newpage

\ \thispagestyle{empty}

\newpage

%********************************************

% stran 3 med uvodnimi listi
\thispagestyle{empty}

Namesto te strani {\bf vstavite} original izdane teme diplomskega dela s podpisom mentorja in dekana ter \v zigom fakultete, ki ga diplomant
dvigne v študent\-skem referatu,  preden odda izdelek v vezavo!

\newpage

%********************************************

% stran 4 med uvodnimi listi je prazna 
\ \thispagestyle{empty}

\newpage

%********************************************

% stran 5 med uvodnimi listi

\thispagestyle{empty}

\vspace{1cm}
\begin{center} 
{\Large \textbf{IZJAVA O AVTORSTVU}}
\end{center}

\begin{center} 
{\Large diplomskega dela}
\end{center}

\vspace{1cm}
Spodaj podpisani/-a \hspace{0.5cm} Ime Priimek,

\vspace{0.5cm}
z vpisno številko \hspace{0.5cm} xxxxxxx,

\vspace{1cm}
sem avtor/-ica diplomskega dela z naslovom:
   
\vspace{0.5cm}
Naslov diplomskega dela

\vspace{1.5cm}
S svojim podpisom zagotavljam, da:
\begin{itemize}
	\item sem diplomsko delo izdelal/-a samostojno pod mentorstvom 
	
	prof. [doc.] dr. Ime Priimek
	
	in somentorstvom 
	
	prof. [doc.] dr. Ime Priimek
	
	\item	so elektronska oblika diplomskega dela, naslov (slov., angl.), povzetek (slov., angl.) ter ključne besede (slov., 			angl.) identični s tiskano obliko diplomskega dela
	\item soglašam z javno objavo elektronske oblike diplomskega dela v zbirki ''Dela FRI''.
\end{itemize}

\vspace{1cm}
V Ljubljani, dne xx.xx.20xx \hspace{1cm} Podpis avtorja/-ice:

\newpage 

%********************************************

% stran 6 med uvodnimi listi je prazna pri dvostranskem tiskanju
\ \thispagestyle{empty}

\newpage
\ \thispagestyle{empty}

%********************************************

% stran 7 med uvodnimi listi

\chapter*{Zahvala}

\thispagestyle{empty}

Na tem mestu se diplomant zahvali vsem, ki so kakorkoli pripomogli k uspešni izvedbi diplomskega dela.
\\**Zahvala podjetju Digital School D.O.O, za pomoč pri analizi učenja otrok z uporabo umetne inteligence. **


\newpage

%********************************************

% stran 8 med uvodnimi listi je prazna pri dvostranskem tiskanju
\ \thispagestyle{empty}

\newpage

%********************************************

\renewcommand\thepage{} 
\tableofcontents 
\renewcommand\thepage{\arabic{page}}

\thispagestyle{empty}


%********************************************

\chapter*{Seznam uporabljenih kratic in simbolov}

\thispagestyle{empty}

Seznam uporabljenih kratic in simbolov, ki morajo biti enotni v celotnem delu, ne glede na označevanje v uporabljenih virih.

%\cleardoublepage

\clearpage{\pagestyle{empty}\cleardoublepage}

%********************************************
%zacno se glavni listi, ki so numerirani z arabskimi stevilkami

\setcounter{page}{1}
\pagenumbering{arabic}

\chapter*{Povzetek}

\addcontentsline{toc}{chapter}{Povzetek}

Povzetek naj posreduje bralcu kratko vsebino dela. Zajema naj namen dela, področje, na katerega se delo nanaša,
uporabljene metode, poglavitne rezultate dela, zaključke in priporočila. 
Povzetek naj ne obsega več kot eno stran, obi\v cajno ima le 200 do 300 besed. Napiše se povsem na koncu,
ko je že jasna vsebina vseh ostalih poglavij.

Ta dokument vsebuje navodilo za izdelavo diplomskega dela v obliki in strukturi, ki je v teh navodilih predpisan za
pisanje diplomskih nalog. Struktura dokumenta je namenjena obojestranskemu tiskanju, kjer se novo poglavje vedno za\v cne na lihi strani.
V dejanski diplomi poglavja in podpogla\-vja  obi\v cajno niso tako kratka kot v teh navodilih.

Za oblikovanje tega dokumenta je bil uporabljen sistem \LaTeX.
Ve\v c o \LaTeX-u lahko izve\v s na spletni strani \texttt{http://www.ctan.org/}.
Kandidati, ki bodo svoje diplomsko delo oblikovali s pomo\v cjo
\LaTeX-a, lahko izvorno kodo tega dokumenta neposredno uporabijo kot vzorec za pisanje svoje diplomske naloge.

\vspace{1.3cm}
\noindent
{\large \bf Ključne besede:}

\vspace{0.5cm}
\noindent
diploma, mentor, zagovor, podaljšanje, pisanje, struktura


\chapter*{Abstract}

\addcontentsline{toc}{chapter}{Abstract}

Povzetek naj bo napisan v angleškem jeziku.

\vspace{1.3cm}
\noindent
{\large \bf Key words:}

\vspace{0.5cm}
\noindent
Ključne besede v angleškem jeziku.


%********************************************

\chapter{Uvod}
Asistenti na področju umetne inteligence so postali zelo pogosto uporabljeno orodje na področju programskega inženirstva, ki pomagajo tako začetnikom napisati preprosto kodo, kot tudi izkušenejšim programerjem pri zahtevnejših nalogah. Po anketi, ki jo je sestavila spletna stran StackOverflow *leto podatka*, kar 44\% programerjev redno uporablja orodja umetne inteligence v razvojnem procesu, dodatnih 25\% pa to načrtuje v bližnji prihodosti (n= 89 184) *citat*. Cilj diplomskega dela je preučiti različne asistente na področju programiranja z umetno inteligenco. Uporabo asistentov na področju umetne inteligence pri učenju programiranja bom tudi preučil na več skupinah otrok, ki se učijo programiranja v programskih jezikih python ter javascript.

\section{Asistenti na področju umetne inteligence} 
 

\section{Hitro programiranje}

\chapter{Pregled področja}
Splošno o pregledu področja
\section{Programiranje s ChatGPT}

\section{Programiranje z GithubCopilot}


\section{Ugotovitve in primerjava}

\chapter{Preučevanje hitrega programiranja na skupini otrok}
\section{Načrtovanje}


\chapter{Prijava diplomskega dela}

\section{Obrazec za prijavo} 
Študent, ki je opravil vse izpite in druge, s študijskim programom predpisane obveznosti, se prijavi k diplomskemu delu s posebnim
obrazcem. 

\section{Kdo izdaja in sprejema obrazce} 
Obrazce za prijavo diplomskega dela izdajajo in sprejemajo v študentskem referatu. 
Tam je po potrebi na voljo tudi obrazec za predčasno oddajo diplom\-skega dela.

\section{Roki za prijavo in oddajo} 
Študent mora oddati prijavo diplomskega dela od 1. do 5. v mesecu, rok pa mu začne teči 15. v mesecu, 
v katerem je oddal prijavo, oz. 5. v mesecu za mesec april.

\section{Kdaj diplomskih del ni mogoče prijavljati} 
Diplomskih del ni mogoče prijavljati v maju, juniju, juliju in avgustu.



\chapter{Navodila za opravljanje diplomskega dela}

\section{Viri}
Za kakovostno diplomsko delo je pomembna uporaba vseh razpoložljivih doma\-čih in tujih strokovnih ter znanstvenih virov.

\section{Predčasen zagovor diplomskega dela}
Diplomant, ki je pri diplomskem delu posebno uspešen in ga zaključi pred rokom, lahko s pristankom mentorja zaprosi za 
predčasen zagovor diplomskega dela. Prošnjo za predčasen zagovor podpišeta na posebnem obrazcu mentor in diplomant. Prošnje sprejema študentski
referat FRI, predčasen zagovor pa odobri prodekan za pedagoško dejavnost.

\section{Nesoglasja med kandidatom in mentorjem}
Če pride med opravljanjem diplomskega dela do nesoglasja med kandidatom in mentorjem ali somentorjem, 
kar onemogoči ustvarjalno sodelovanje, ima kandidat na podlagi sklepa Komisije za študijske zadeve pravico do zame\-nja\-ve mentorja ali
somentorja. Zaradi istih razlogov in po enakem postopku lahko tudi mentor ali somentor odklonita mentorstvo oziroma somentorstvo. V takšnem
primeru je potrebno z navedbo razlogov pisno zaprositi Komisijo za študijske zadeve, ki na podlagi sklepa odobri zamenjavo in določi novega
mentorja ali somentorja. Kandidat ima pravico zamenjave mentorja uveljaviti le enkrat.



\chapter{Navodila za pisanje}

\section{Razgledanost}
Pri pisanju diplomskega dela izkazuje kandidat poleg strokovne uspo\-sob\-lje\-nos\-ti še splošno razgledanost.

\section{Jezik diplomskega dela}
Diplomsko delo mora biti napisano v slovenskem jeziku in mora biti jezikovno neoporečno. 
Priporočamo, da pisni izdelek pred oddajo pregleda lektor. Obsega naj najmanj 20 strani strokovnega besedila. Stro\-kov\-no besedilo ni
li\-te\-rar\-ni tekst in zanj veljajo drugačna pravila. Stro\-kov\-na besedila morajo biti nedvoumna, zato naj bodo stavki kratki, preprosto
razumljivi in naj vsebujejo čimmanj podredniških zvez. Ena in ista stvar naj bo vedno poimenovana z istim imenom. Samostalniški prilastki so v
slovenščini na desni strani osebka ali predmeta (SQL query $\rightarrow$ poizvedba SQL, API interface $\rightarrow$ vmesnik API), zato naj
velja posebna pozornost prevodom iz angleščine. S pisanjem strokovnih besedil se razvija tudi slovenska strokovna terminologija. Vsak avtor
mora upo\-šte\-va\-ti že uveljavljene slovenske prevode strokovnega izrazja, pri novem stro\-kov\-nem izrazju, ki se pojavlja v tujih jezikih,
še zlasti v angleščini, pa se mora potruditi za iskanje ustreznih slovenskih prevodov. 

\section{Papir, pisave in vezava}
Besedilo mora biti napisano na belem papirju formata A4.
Okvirna postavitev besedila na omenjenem formatu je podana v tabeli \ref{tabela_mere}.

\begin{table}[htb]
\begin{center}
\begin{tabular}{|l|l|}\hline
zgornji rob	 & 20 mm (nad pagino vivo, če je ta uporabljena)\\\hline
spodnji rob	 & 30 mm\\\hline
notranji rob & 30 mm\\\hline
zunanji rob	 & 20 mm\\\hline
\end{tabular}
\end{center}
\caption[Okvirna postavitev besedila.]{Okvirna postavitev besedila na formatu papirja A4.}
\label{tabela_mere}
\end{table}

Priporoča se pisava ``Times New Roman'' ali pisava ``Computer Modern'', obe velikosti 12. 
Razmak med vrsticami naj bo običajen, tisk pa po možnosti obojestranski.

Celoten izdelek naj bo vezan v platno ali drug ustrezen material. 
Barva tega materiala naj bo črna ali temno modra. 
Na hrbtu vezanega dela naj bo napisano ime in priimek kandidata in ``Diplomsko delo''.

\section{Platnica diplomskega dela}
Na platnici mora biti: 
\begin{itemize}
\item	naziv univerze in fakultete z velikimi črkami, oddaljen ca. 30 mm od zgornjega roba; 
\item	ime in priimek kandidata in pod njim naslov dela;
\item	oznaka, za katero vrsto diplomskega dela gre (univerzitetno ali visoko\-šol\-sko strokovno), napisana prav tako z velikimi črkami;
\item	spodaj na sredini ``Ljubljana, letnica'', oddaljena ca. 30 mm od spodnjega roba. 
\end{itemize}
Diplomsko delo lahko razdelimo na uvodne liste in glavne liste.

\section{Uvodni listi}
Uvodni listi si sledijo po naslednjem vrstnem redu: 
\begin{itemize}
\item	
naslovna stran je enaka platnici, le da ima še navedbo mentorja:
naziv univerze in fakultete, ime in priimek diplomanta, 
naslov teme diplom\-ske\-ga dela (enak kot v originalu izdane teme), oznaka za katero vrsto diplomskega dela gre, navedba mentorja ter kraj in
letnica; 
\item	original izdane teme diplomskega dela, ki ga diplomant dvigne v študent\-skem referatu, preden odda izdelek v vezavo;
\item izpolnjena izjava o avtorstvu in soglasje, ki omogoča objavo elektronske oblike v zbirki 'Dela FRI'; 
\item	zahvala, v kateri se diplomant zahvali vsem, ki so kakorkoli pripomogli k uspešni izvedbi diplomskega dela; 
\item	morebitno posvetilo;
\item	kazalo, ki obsega oštevilčene naslove poglavij in podpoglavij vključno s številkami strani. 
Za oštevilčenje poglavij se uporablja večnivojsko desetiško številčenje (kot npr. v tem navodilu);
\item	seznam uporabljenih kratic in simbolov, ki morajo biti enotni v celotnem delu, ne glede na označevanje v uporabljenih virih. 
\item uvodne strani nimajo označb strani.
\end{itemize}

\section{Glavni listi}
Od tu dalje so strani označene s številkami, ki naj bodo na zunanjem robu strani zgoraj. 
Delo obsega naslednje dele: povzetek, uvod (1. poglavje), glavni del, ki je smiselno razdeljen na več poglavij (2., 3., \ldots\ poglavje),
sklepne ugotovitve (zadnje poglavje), morebitne priloge, seznam slik in tabel, seznam uporabljenih virov (literatura). 

\subsection{Povzetek}
Povzetek naj posreduje bralcu kratko vsebino dela. Zajema naj namen dela, področje, na katerega se delo nanaša,
uporabljene metode, poglavitne rezultate dela, zaključke in priporočila.  
Povzetek naj ne obsega več kot eno stran. Napiše se povsem na koncu, ko je že jasna vsebina vseh ostalih poglavij.

\subsection{Uvod}
Uvod v diplomsko delo ima namen, da uvede bralca v tematiko dip\-lom\-ske\-ga dela. Za izhodišče pri pisanju uvoda služijo zahteve
in cilji naloge, ki so navedeni v temi diplomske naloge, definirani s strani mentorja. Sledi podrobnejši opis problemskega področja, okolja v
katerem so problemi pojavljajo, in identifikacija problemov, ki jih je potrebno razrešiti. Marsikateri problem in njegova rešitev je že opisan
v literaturi, zato je potrebno pregledati vire in oceniti njihovo koristnost pri izdelavi naloge. Če že obstajajo kakšne podobne rešitve, je
potrebno ugotoviti do kolikšne mere se dajo uporabiti, oziroma kakšne modifikacije bi bile potrebne za njihovo uporabo.
Na koncu uvoda naj bo podan tudi kratek pregled vsebine posameznih poglavij.

\subsection{Glavni del}
V glavnem delu diplomskega dela opiše kandidat potek izdelave svojega dela, 
pri čemer je predhodno seveda dokončal praktični del naloge.  V njem razloži, katere poti je ubral za dosego cilja, in sistematično opiše
opravljeno delo.

Kandidat oblikovno razdeli gradivo smiselno na poglavja, podpoglavja in morebiti še na razdelke, ki jih oštevilči 
(na primer 4.6.3.). 

Slik, ki skrajšujejo besedilo, ali pripomorejo k razumljivosti, naj bo čim več. Slike ali fotografije morajo biti
oštevilčene in citirane v besedilu (npr. slika \ref{slika_seznam}) ter podnaslovljene tako, da je razvidno, kaj predstavljajo. V besedilo so
vstavljene približno tam, kjer se nanje sklicujemo. Slike naj bodo pregledne in naj prikažejo le najpotrebnejšo informacijo. 

\begin{figure}[htb]
\centerline{\psfig{figure=image1,width=13cm}}
\caption[Krožni dvosmerni seznam.]{Krožni dvosmerni seznam za n=15. Kazalec prvi služi kot vstopna to\v cka.}
\label{slika_seznam}
\end{figure}

Formule in enačbe se oštevilči z zaporedno številko v oklepaju na desni strani formule, npr. enačba (\ref{enacba_r}), 
in se tako nanje tudi sklicuje. Kadar za določen strokovni termin ni splošno sprejetega domačega izraza, se navede prvič, ko se slovenski izraz
pojavi, v oklepaju tudi originalni izraz, povzet iz uporabljene literature.

\begin{equation}
r = \pi R_1 (r) \bowtie \pi R_2 (r) \bowtie \cdots \bowtie  \pi R_n  (r)
\label{enacba_r}
\end{equation}

\subsection{Sklepne ugotovitve}

Sklepne ugotovitve naj prikažejo oceno o opravljenem delu in povzamejo težave, na katere je naletel kandidat. Kot rezultat dela
lahko navede ideje, ki so nastale med delom, in bi lahko bile predmet novih raziskav.

\subsection{Priloge}
Priloge (slike, diagrami, algoritmi, načrti), 
če so potrebne, kandidat izdela kot posebna poglavja (dodatki), ki jih zaradi preglednosti ni smiselno vključiti v glavni del naloge. Vsi
dodatki morajo biti naslovljeni in oštevilčeni, običajno z velikimi tiskanimi črkami. 

Na tem mestu je priporočljivo navesti tudi seznama slik in tabel.

\subsection{Viri}
Vire navede kandidat po abecednem vrstnem redu priimkov avtorjev. Navajajo naj se le tisti viri, na katere se
kandidat v besedilu sklicuje! Dolg seznam ni dokaz, da ima kandidat tudi tak pregled čez literaturo. Za navajanje virov naj se uporabi
slovenska inačica formata IEEE. V tem formatu se viri (članek v zborniku konference,  knjiga, članek v reviji in spletna stran) navedejo  kot
je to prikazano v poglavju Literatura (stran \pageref{stran_literatura}). Sklicevanje na vir se v besedilu naloge označi z zaporedno številko
vira v oglatem oklepaju, npr. \cite{peytonjones93} oziroma \cite{trucco98,wadler89,IEEE}, kadar se na istem mestu sklicuje na več
različnih virov.  Zglede za druge vrste virov je mogoče dobiti v revijah IEEE, ki so na voljo v knjižnici, ali na spletnem naslovu
\cite{IEEE}.

\subsection{Pisanje v tujem jeziku}
Kandidatu se dovoli pisanje diplomskega dela v enem izmed svetovnih jezikov (npr. angleščini) na osnovi utemeljene prošnje. Naslovna stran
(platnica) diplomskega dela je napisana v slovenskem jeziku. V naslednjem zaporedju si sledijo: prva stran diplomskega dela v tujem jeziku,
prva stran diplomskega dela v slovenskem jeziku in original izdane teme diplomskega dela v tujem jeziku. Povzetek diplomskega dela v slovenskem
jeziku mora obsegati 3 do 5 strani. 


\chapter{Oddaja in zagovor diplomskega dela}

\section{Roki za zagovor}
Roke za oddajo in pogoje za zagovor diplomskega dela določata ``
Pravilnik o preverjanju in ocenjevanju znanja ter izpitnem redu FRI''
in ``Statut univerze''.

\section{Podalj\v sanje roka}
Ko je kandidatov mentor zadovoljen z dosežki kandidata in ko le-ta do\-kon\-ča diplomsko delo, sledi zagovor. Če ob izteku roka za izdelavo
diplomskega dela mentor ni zadovoljen s kandidatovimi dosežki, mora kandidat zaprositi za podaljšanje roka oddaje in delo dopolniti glede na
zahteve mentorja.

V primeru, da kandidat tudi v podaljšanem roku ne odda diplomskega dela, roka ne more več podaljšati in mora zaprositi za izstavitev nove teme.

\section{Kdaj tema zapade}
Diplomantu, ki ne odda v roku diplomskega dela in ne zaprosi za njegovo podaljšanje, izda fakulteta ugotovitveni sklep, da je tema zapadla. Za
izstavitev nove teme diplomskega dela mora kandidat s pisno vlogo zaprositi Komisijo za študijske zadeve.

\section{\v Stevilo izvodov}
Kandidat odda dva oz. tri enake vezane izvode diplomskega dela, enega v študentski referat, drugega mentorju oziroma tretjega somentorju, če je
bil ta imenovan. Vezan izvod, ki ga študent odda v študentski referat, mora imeti dodan elektronski medij (CD), ki vsebuje:
\begin{itemize}
	\item celoten izvod diplomskega dela v elektronski obliki v formatu ....pdf (izvod ne sme biti na noben način zaščiten).
	\item naslov (slov., angl.), povzetek diplomskega dela (slov., angl.) ter ključne besede (slov., angl.) oblikovane  z ustreznim urejevalnikom besedil in shranjene v formatu .doc ali .rtf. Podatki se uporabijo za vnos v digitalno zbirko ePrints, zato morajo biti napisani z malimi črkami v povezavi s potrebnimi velikimi začetnicami in kraticami, ki so zapisane z velikimi črkami. 
\end{itemize}

Študent odda v študentski referat tudi podpisan obrazec, ki vključuje izjavo o avtorstvu in soglasje, ki omogoča objavo elektronske oblike v zbirki 'Dela FRI'. Obrazec je na spletni strani FRI: \textit{Izjava o avtorstvu diplomskega dela.doc}. 


\section{Komisija za zagovor diplomskega dela}
Kandidat zagovarja svoje delo na diplomskem izpitu pred komisijo, ki jo se\-stav\-lja mentor in vsaj dva učitelja. Zagovor diplomskega dela je
javen. Zagovor vodi predsednik komisije za oceno in zagovor diplomskega dela. 

\subsection{Način zagovora}
Predstavitev diplomskega dela je ustna, kandidat pa ne sme brati vna\-prej pripravljenega besedila. To mu je lahko samo v pomoč. Dovoljeno je
pre\-brati številčene podatke in citate. Kandidat naj predstavitev diplomskega dela popestri z ilustrativnim prikazom dosežkov. Pri tem sme
uporabiti vsa primerna sredstva, vključno z multimedijskimi. O obliki predstavitve se dogovori z mentorjem. Za zagotavljanje tehničnih
pripomočkov se diplomant dogovori z mentorjem ali z referentom v štu\-dent\-skem referatu vsaj en dan pred zagovorom. Kadar ta sredstva niso
zadostna oziroma primerna, jih diplomant lahko z mentorjevim soglasjem dopolni z laboratorijskimi ali lastnimi. V tem primeru jih mora takoj po
zagovoru pospraviti, da se v diplomski sobi lahko začne nov zagovor. O času, v katerem je diplomska soba na razpolago diplomantu, se ta
dogovori v študentskem referatu.

\subsection{\v Cas za zagovor}
Zagovor diplomskega dela začne kandidat s kratko, praviloma 10 minutno predstavitvijo svojega dela. Kandidat mora biti sposoben, da v razmeroma
kratkem času poda članom komisije in drugim poslušalcem poglavitno vsebino svojega dela. Uvodoma naj razloži, kaj je predmet njegovega dela,
katerih problemov se je lotil, kakšne so bile zahteve in kakšne vire je imel na voljo za njihovo rešitev. Sledi opis reševanja problemov v
skladu s podanimi specifikacijami, npr. izdelava programske opreme, izdelava strojne opreme, razvoj algoritmov, metod ali metodologij itd. V
zaključku kandidat kritično oceni rezultate svojega dela ter poda ideje in smernice za njegovo nadaljevanje.

\subsection{Vpra\v sanja}
Po ustni predstavitvi člani komisije postavijo kandidatu vprašanja. Kandidat odgovori na vprašanja iz celotne tematike diplomskega dela, iz
usmerjenega znanja dodiplomskega študija, na katerega se opira diplomsko delo, in iz splošnega temeljnega znanja računalništva in informatike.
Na vprašanja mora kandidat odgovoriti jasno, kratko in suvereno.

\subsection{Ocena diplomskega dela in ocena zagovora}
Po končanem zagovoru se komisija za oceno in zagovor diplomskega dela oddalji in oceni diplomsko delo in zagovor. Po vrnitvi v prostor zagovora
predsednik komisije za oceno in zagovor ustno sporoči oceno diplomskega dela, oceno zagovora diplomskega dela in končno oceno diplomskega
izpita. 

\subsection{Pritožba}
Če se diplomant ne strinja s katero od ocen (dela ali zagovora), se lahko pisno pritoži dekanu. Pritožbo mora oddati v dekanat fakultete v roku
24 ur po zagovoru. 

\newpage

%********************************************

\appendix

%\addcontentsline{toc}{chapter}{\protect Dodatki}

\chapter{Kaj so priloge ali dodatki}

Priloge (slike, diagrami, algoritmi, načrti), 
če so potrebne, kandidat izdela kot posebna poglavja (Dodatek A, Dodatek B, \ldots), ki jih zaradi preglednosti ni smiselno vključiti v glavni
del naloge. Vsi dodatki morajo biti naslovljeni in oštevilčeni, običajno z velikimi tiskanimi črkami. 

\newpage

\addcontentsline{toc}{chapter}{Seznam slik}
\addtocontents{toc}{\protect\vspace{-2ex}}
\listoffigures

\newpage

\addcontentsline{toc}{chapter}{Seznam tabel}
\listoftables

%\listofalgorithms


%********************************************

\newpage

\bibliographystyle{slplainurl}
\addcontentsline{toc}{chapter}{Literatura}
\label{stran_literatura}
\bibliography{diploma} 


\end{document}
